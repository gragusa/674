\documentclass{tex/tufte-handout}
% Copyright © 2013 Gray Calhoun.

% Permission is granted to copy, distribute and/or modify this
% document under the terms of the GNU Free Documentation License,
% Version 1.3 or any later version published by the Free Software
% Foundation; with no Invariant Sections, no Front-Cover Texts, and no
% Back-Cover Texts.  A copy of the license is included in the file
% LICENSE.tex and is also available online at
% <http://www.gnu.org/copyleft/fdl.html>.

\usepackage{
  amsmath,
  bibentry,
  booktabs,
  chngcntr,
  enumitem,
  fancyhdr,
  fancyvrb,
  graphicx,
  ifdraft,
  ragged2e,
  sourcesanspro,
  subfig,
  tabularx,
  textcomp,
  url,
  verbatim,
  xcolor,
}

\usepackage[T1]{fontenc}
\usepackage[utf8x]{inputenc}
\usepackage[bitstream-charter]{mathdesign}
\usepackage[letterspace=35]{microtype}
\usepackage[osf]{sourcecodepro}
\usepackage[round,comma]{natbib}
\usepackage[top=1in,bottom=1.25in,left=1.25in,right=1.25in]{geometry}

% Change to ragged-right without inline math breaks; this helps avoid
% awkward line breaks or extensions into the right margin in math or
% code.
\RaggedRight
\relpenalty=10000
\binoppenalty=10000
\setlength{\parindent}{1em}

\renewcommand{\FancyVerbFormatLine}[1]{\hspace{\leftmargini}#1}
\fvset{frame=leftline,rulecolor=\color{lightgray}}

\urlstyle{same}
\DeclareUrlCommand\url{%
\def\UrlLeft{\guillemotleft}%
\def\UrlRight{\guillemotright}%
\urlstyle{same}}

% This will make copy and pasting easier, and will prevent the
% abomination that is monospace font ligatures.  Sorry, everyone who
% really likes ligatures.  Don't worry, the Bitstream Charter font
% has very, very few ligatures anyway.
\DisableLigatures{family = tt*}
\frenchspacing
\setcounter{secnumdepth}{2}
\setcounter{tocdepth}{1}
\counterwithout{section}{chapter}
\renewcommand\thesection{\arabic{section}}

\ifoptionfinal{
  % If the 'final' option is set, \try[etc] just includes or inputs
  % the file
  \newcommand{\tryincludegraphics}[1]{\includegraphics{#1}}
  \newcommand{\tryinput}[2][{}]{\input{#2}}
  \newcommand{\tryinclude}[2][{}]{\include{#2}}
}{
  % If the 'final' option is not set, \try[etc] includes or inputs the
  % file if it exists, but continues if it doesn't.
  % Based on a TeX stackexchange answer:
  % http://tex.stackexchange.com/a/39983/9423
  \newcommand{\tryincludegraphics}[1]%
  {\IfFileExists{#1}{\includegraphics{#1}}%
    {\framebox{Figure missing. See README for help.}}}
  \newcommand{\tryinput}[2][{}]{\IfFileExists{#2}{\input{#2}}{#1}}
  \newcommand{\tryinclude}[2][{}]{\IfFileExists{#2}{\include{#2}}{#1}}
}

\newtheorem{thm}{Theorem}[section]
\newtheorem{defn}{Definition}[section]
\newtheorem{ex}{Example}[section]
\newtheorem{hw}{Problem}[section]
\newtheorem{asmp}{Assumption}[section]

\renewcommand{\labelitemi}{{$\circ$}}
\renewcommand{\labelitemii}{\footnotesize$\circ$}
\renewcommand{\labelitemiii}{\textperiodcentered}
\renewcommand{\labelitemiv}{\footnotesize\textperiodcentered}

\newcommand{\OECD}{\allcaps{OECD}}
\newcommand{\IJF}{\allcaps{IJF}}

%%% Local Variables:
%%% mode: latex
%%% TeX-master: "../macroeconometrics"
%%% End:

\newcommand{\version}{0.2.3}

\title{Notes on Macroeconometrics}
\author{Econometrics Free Library Project}

\begin{document}
\maketitle

\bigskip\noindent%
Copyright © 2013, the authors of the \textit{Notes on
  Macroeconometrics} textbook.  A complete author list should be
included as an appendix to this document.

Permission is granted to copy, distribute and/or modify this document
under the terms of the GNU Free Documentation License, Version 1.3 or
any later version published by the Free Software Foundation; with no
Invariant Sections, no Front-Cover Texts, and no Back-Cover Texts.  A
copy of the license should be included with this document and is also
available online at \url{http://www.gnu.org/copyleft/fdl.html}.

This text was produced as part of the Econometrics Free Library
Project.  The project's goal is to produce high-quality free (and
open-source) Econometrics textbooks and reference material.  More
information about this project is available at its homepage,
\url{http://www.econometricslibrary.org}, including information on how
to participate.  This text is typeset in \XeLaTeX\ and there is a link
to the source of the document at the project's homepage as well.

\tableofcontents

\section{A quick comment about these notes}
This text started as a set of lecture notes that Gray Calhoun wrote to
himself to teach half of a semester-long Macroeconometrics PhD
elective at Iowa State in 2013.  He (I) had no plans to disseminate
them until the course was over.  As a first step, the notes have been
converted to a \XeLaTeX\ document, but there is a lot of work left
before they would be useful to anyone trying to learn this material.

% Copyright © 2013, Gray Calhoun.  Permission is granted to copy,
% distribute and/or modify this document under the terms of the GNU Free
% Documentation License, Version 1.3 or any later version published by
% the Free Software Foundation; with no Invariant Sections, no
% Front-Cover Texts, and no Back-Cover Texts.  A copy of the license is
% included in the section entitled "GNU Free Documentation License."

\part{Introductory lectures on time series and ARMA modeling}

Textbook material:
\begin{itemize}
\item Hayashi:
  \begin{itemize}
  \item Chapter 2 (2.2 especially)
  \item Chapter 6
  \end{itemize}
\item Greene 6th edition:
  \begin{itemize}
  \item Chapter 19
  \end{itemize}
\item Greene 7th edition:
  \begin{itemize}
  \item Chapter 20
  \end{itemize}
\end{itemize}

Other books that students may want to consult:
\begin{itemize}
\item Peter Brockwell and Richard Davis (1991), \emph{Time series:
    theory and methods, second edition}.
\item Jim Hamilton (1994), \emph{Time series analysis}.
\end{itemize}

\section{Basic definitions}

\begin{description}
\item[Stochastic process:]
A stochastic process is a collection of random variables (or
random vectors) indexed by a parameter $t$ indicating time.
\end{description}

Want to discuss how this can be viewed as a sequence of random
variables or as a random function of $t$ (sample path).

Often we need to deal with \textbf{triangular stochastic arrays} in a
formal sense:

\begin{description}
\item[Motivation for stochastic array:]
  Suppose $\{y_t; t=1,...,n\}$ is a stochastic process where each
  $y_t$ is i.i.d standard normal. Let $s^2_n$ be the sample variance
  of this process. Is $z_t = (y_t - \bar y) / s$ a stochastic process?
  (no)

  So we introduce two sets of indices and write $z_{t,n} = (y_t - \bar
  y) / s_n$; where $n = 1,2,...$ and $t = 1,...,n$.  Now $\{z_{t,n}\}$
  is a stochastic array.
\item[Strict stationarity:]
  The stochastic process $\{y_t\}$ is a \textbf{strictly stationary
    time series} if, for all finite s, t, and k, we have
  \[(y_s,...,y_{s+k}) =^d (y_t,...,y_{t+k})\]

  Example of strict stationarity: suppose that $y_t$ is i.i.d.; then
  it is strictly stationary.

  Another example: suppose that, for any $t$, $(y_t, y_{t+1},
  y_{t+2})$ is distributed $N(μ,Σ)$, and that $y_t$ and $y_{t+3}$ are
  independent for all $t$; then $y_t$ is obviously strictly
  stationary.

  A counter-example. Suppose that $Δy_t$ is i.i.d. $N(0,1)$ and $y_0$
  is zero. Then $E y_i = 0$ for all $i$, but
  \[var(y_i) = \sum_{t=1}^i var(Δy_t) = i\]

  so the series is not stationary.

  Other examples too (e.g. breaks)
\end{description}

Strict stationary is usually ``unnecessarily'' restrictive; usually
asymptotic properties come from the first two moments, so we have a
variation of stationarity that applies to the first two moments.

\begin{description}
\item[Weak stationarity:] 
  The stochastic process $\{y_t\}$ is a \textbf{weakly stationary time
    series} if
  \[ E y_t = μ ∀t \] and 
  \[cov(y_t, y_{t+k}) = cov(y_s, y_{s+k}) ∀ \ s,t, k\] informally, the
  first two moments do not depend on $t$.
\item[Autocovariance, Autocorrelation:]
  The \textbf{autocovariance} function of a weakly stationary time
  series is defined as
  \[Γ(k) = cov(y_t, y_{t-k})\] (univariate or multivariate) The
  \textbf{autocorrelation} function of a weakly stationary time series
  is defined as
  \[Ρ(k) = var(y_t)^{-1} Γ(k)\]

  ``Weak dependence'' requires that $Γ(k) → 0$ as $k → ∞$.

  Obvious properties if $Γ(\cdot)$ is the autocovariance function of
  $y_t$:
  \begin{itemize}
  \item $var(y_t) = Γ(0)$
  \item $Γ(k) = Γ(-k)'$ for all $k$.
  \end{itemize}

  Properties if $y_t$ is univariate
  \begin{itemize} 
  \item $|γ(k)| ≤ γ(0)$ for all $k$.
  \item The autocovariance function is \emph{nonnegative definite}.
  \end{itemize}
\item[Nonnegative definite (extends concept for matrices):]

  A real-valued function f is nonnegative definite if, for any natural
  number $n$, any $a ∈ R^n$ and any $t ∈ Z^n$, we have
  \[\sum_{i=1}^n \sum_{j=1}^n a_i a_j f(t_i - t_j) ≥ 0.\]
\end{description}

In this part of the class, we are going to assume that the series are
already stationary (weak, usually) and are weakly-dependent.

\begin{itemize}
\item Statistical agencies typically provide seasonally-adjusted data
  and the data are usually aggregated enough that holidays, etc. are
  ignored.
\item Accounting for trends and permanent shocks is a big deal in
  macro; I'll try to add notes in the future.
\end{itemize}

\section{VAR and VMA processes}

\begin{itemize}
\item As a starting point, we are going to build dynamic processes as
  linear functions of ``white noise.''
\item Approach is analogous to what we did with OLS: we will talk
  about a statistical model that seems pretty narrow, but show how to
  get economic content out of it and why it might hold approximately
  in a general setting.
\end{itemize}

\begin{description}
\item[White noise; i.e. $WN(0,σ^2)$:]
  A stochastic process $\{e_t\}$ is $WN(0, σ^2)$ if each $e_t ∼ (0,
  σ^2)$ and $cov(e_t, e_s) = 0$ when $s ≠ t$.

  Obviously, the same definition applies when the $e_t$ are random
  vectors.
\end{description}

To build $y_t$ from a white noise process $e_t$, there are two obvious
approaches:
\begin{itemize}
\item Let $y_t$ depend on $e_t$ as well as past $e_s$ ($s < t$).
\item Let $y_t$ depend on past $y_s$ ($s < t$).
\item We're going to (for now) build $y_t$ from linear functions of
  $e_t$ and past $y_t$.
\end{itemize}

\begin{description}
\item[(Vector) Moving average process of order $q$:]

  $y_t$ is a VMA(q) if it is a stationary solution to the equation
  \[y_t = μ + e_t + \sum_{i=1}\^q θ_i e_{t-i}\] where $e_t$ is $WN(0,
  Σ)$, $μ$ is a constant $k$-vector, and $θ_i$ is a $k$ by $k$ matrix.

  Finding the moments of an $MA(q)$ process is easy and it is obvious
  that it is covariance stationary.

  \begin{itemize}
  \item $E y_t = μ + E e_t + \sum_{i=1}^q θ_i E e_{t-i} = μ$.
  \item $var(y_t) = var(e_t) + \sum_{i=1}^q θ_i var(e_{t-i}) θ_i' =
    \sum_{i=0}^q θ_i Σ θ_i'$ where $θ_0 = 1$.
  \item The autocovariances are straightforward too. Assume zero mean,
    then: 
    \[E y_t y_{t-j}' = E \sum_{i=0}^q θ_i e_{t-i} \sum_{k=0}^q e_{t-j-k}' θ_k' 
    = \sum_{i=j}^q θ_i E e_{t-i} e_{t-i}' θ_{i-j}'
    = \sum_{i=1}^q θ_i Σ θ_{i-j}'\] 
    This is zero if $i ≥ q$ (note that if $e_t$ is Normal, $MA(q)$
    processes are $q$-dependent).
  \end{itemize}
\item[(Vector) Autoregressive process of order p:]
  $y_t$ is a VAR(p) if it is a stationary solution to the difference
  equation 
  \[y_t = μ + \sum_{i=1}^p φ_i y_{t-1} + e_t\]
  where $e_t$ is $WN(0, Σ)$, $μ$ is a constant, and $φ_i$ are $k$ by
  $k$ matrices

  To get the mean, suppose that the process is covariance
  stationary. Then $E y_t = E y_{t-i}$ and so
  \[E y_t = μ + \sum_{i=1}^p φ_i E y_t + 0.\]

  If $I - \sum_{i=1}^p φ_i$ is invertible, then we have
  \[E y_t = (I - \sum_{i=1}^p φ_i )^{-1} μ\]
  (invertibility of this matrix turns out to be a necessary condition
  for covariance stationarity).

  For the autocovariances (which will give us the variance), assume
  0-mean, covariance stationary. Since
  \[y_t = \sum_{i=1}^p φ_i y_{t-i} + e_t,\]
  we can post-multiply by $y_{t-j}$ to get
  \[y_t y_{t-j}' = \sum_{i=1}^p φ_i y_{t-1} y_{t-j}' + e_t y_{t-j}'\]
  and (assuming $e_t$ and $y_{t-j}$ are uncorrelated)
  \[E y_t y_{t-j}' = \sum_{i=1}^p φ_i E y_{t-i} y_{t-j}'.\]

  This gives a recursive definition for the autocovariances. To get
  the initial conditions, we have $(j = 1,…,p)$
  \[Γ(0) = \sum_{i=1}^p φ_i Γ(i) + Σ Γ(j) = \sum_{i=1}^p φ_i Γ(j-i) \]

  Since $Γ(-j) = Γ(j)'$ this gives us a system of equations that can
  be solved for $Γ(0),…,Γ(p)$. Then, for higher-order autocovariances,
  we have the relationship (if $j ≥ p$)
  \[Γ(j) = \sum_{i=1}^p φ(i) Γ(j-i),\]
  which can be solved recursively or explicitly through difference
  equations.

  Note that the autocovariance dies out slowly over time for AR
  processes.
\item[Canonical representation of a VAR(p):]
If we define: 
\begin{align}
  w_t &= (y_t, y_{t-1}, …, y_{t-p+1})' \\
  u_t &= (e_t, 0, …, 0)' \\
  Ψ &= \begin{pmatrix}
    φ_1 & φ_2 & φ_3 & ⋯ & φ_{p-1} φ_p \\
    I & 0 & 0 & ⋯ & 0 & 0 \\
    0 & I & 0 & ⋯ & 0 & 0 \\
    ⋮ \\
    0 & 0 & 0 & ⋯ & I & 0
  \end{pmatrix}
\end{align}
Then we can write a VAR($p$) model as the VAR(1): 
\[w_t = Ψ w_{t-1} + e_t\]
\end{description}

\subsection{Lag polynomials:}

\begin{description}
\item[Lag operator:]

  The lag operator $L$ is defined as the operator s.t. $L y_t =
  y_{t-1}$

  The inverse is $L^{-1} y_t = y_{t+1}$.

  Behaves formally like premultiplication by a constant:
  \begin{itemize}
  \item $c ∙ (L y_t) = (c y_t)$
  \item $a + L y_t = L (a + y_t)$
  \item $L^k y_t = L(⋯ L(L(L y_t)) ⋯) = L(⋯ L(L y_{t-1}) ⋯) = y_{t-k}$
  \end{itemize}
  Lets us write ``lag polynomials'': 
  if $θ(z) = \sum_{i=0}^k θ_i z^i$ then 
  \[θ(L) y_t = \sum_{i=0}^k θ_i y_{t-i}.\]
\end{description}

Notice that we can write the AR(p) process as
\[φ(L) y_{t} = μ + e_t \]
where $φ(L) = I - \sum_{i=1}^p φ_i L^i$.

and the MA(q) as
\[y_t = μ + θ(L) e_t\]
where $θ(L) = I + \sum_{i=1}^q θ_i L^i$.

The notation suggests that we can ``solve'' for $y_t$ or ``solve'' for
$e_t$:
\[y_t = φ(L)^{-1} μ + φ(L)^{-1} θ(L) e_t\]
(making an MA process, possibly MA(∞)) or
\[θ(L)^{-1} φ(L) y_t = θ(L)^{-1} μ + e_t\]
(making an AR process, possibly AR(∞)).

Under some restrictions on the polynomials, we can do this; it's
essentially the same as taking a Taylor-series approximation for the
polynomial $θ(z)^{-1} or φ(z)^{-1}$.

\begin{itemize}
\item Suppose that $y_t$ satisfies $φ(L) y_t = e_t$. If $det φ(z) ≠ 0$
  for all (complex) $z$ s.t. $|z| ≤ 1$, then $y_t$ is covariance
  stationary and can be expressed as the MA(∞):
  \[y_t = φ(L)^{-1} e_t.\]
  \begin{itemize}
  \item This is called a \emph{causal} autoregressive process
  \item \emph{unit root}(s) if $det φ(z) = 0$ for $|z| = 1$; number of
    ``unit roots'' is the order of the solution.
  \item Give AR(1) example
  \item Mention non-causal AR processes as well
  \end{itemize}
\item Suppose that $y_t$ is the MA($q$) process $y_t = θ(L) e_t$. If
  $det θ(L) ≠ 0$ for all complex $z$ s.t. $|z| ≤ 1$, then $y_t$ can be
  expressed as the AR(∞) process: $θ(L)^{-1} y_t = e_t$

  \begin{itemize}
  \item This is called an \emph{invertible} moving average process
  \end{itemize}
\item Both of these results hold even if $φ$ or $θ$ are infinite-order
  polynomials.
\end{itemize}

\subsection{VARMA(p,q) models}

\begin{description}
\item[(V)ARMA(p, q):] $y_t$ is a $VARMA(p,q)$ process if it is the
  stationary solution to the difference equation $φ(L) y_t = μ + θ(L)
  e_t$ where $e_t$ is $WN(0, σ^2)$ and $μ$ is a constant, $φ$ and $θ$
  are polynomials of order $p$ and $q$ such that $φ(0) = θ(0) = I$.

  Note that if $φ(z)$ and $θ(z)$ have the same roots/zeros, so $φ(L) =
  a(L) b(L)$ and $θ(L) = a(L) c(L)$ then we can typically factor out
  and remove $a(L)$, giving a VARMA process with the same dynamics:
  $b(L) y_t = μ^* + c(L) e_t$; so we typically assume that there are
  no common roots

  Same conditions as above apply if we want causality and
  invertiblity.
\end{description}

\subsection{Dynamics:}

For \textbf{univariate} time series, people typically focus on
\begin{itemize}
\item Autocorrelation Function (ACF)
\item Partial Autocorrelation Function (PACF)
\end{itemize}
This holds whether or not the true DGP is thought to be an AR or MA
process.

\begin{description}
\item[ACF:]
  The Autocorrelation function is just $ρ(j)$ viewed as a function of
  $j$; we've already calculated the autocovariance for $AR(p)$ and
  $MA(q)$ processes, so the autocorrelation is just a matter of
  scaling.
\item[PACF:]
  The $j$th partial autocorrelation of a weakly stationary series is
  defined as $corr(y_t, y_{t-j} ∣ y_{t-1},…, y_{t-j+1})$.

  Define this as $α(j)$ for now

  The partial autocorrelation function is $α(j)$ as a function of $j$.

  Analogously to OLS, we know that $α(j)$ is the coefficient $β_j$ on
  $y_{t-j}$ in the equation
  \[y_t = β_0 + β_1 y_{t-1} + β_2 y_{t-2} + ⋯ + β_j y_{t-j} + u_t\]
  where the $β_i$ minimize the population MSE.

  Solution method: $(α(0), α(1)) = (γ(0), γ(1))$. Then it works just
  like for OLS:
  \[ y_t = ( y_{t-1}, y_{t-2},…, y_{t-i} ) β(i) + u_{t,i} \] giving
  \[( y_{t-1},…, y_{t-i} )'y_t = ( y_{t-1},…, y_{t-i} )' ( y_{t-1},…,
  y_{t-i} ) β(i) + ( y_{t-1},…, y_{t-i} )' u_{t,i}\]
  Take expectations and solve to get
  \begin{equation}
    \begin{pmatrix}
      γ(0) & γ(1) & ⋯ & γ(i) \\
      γ(1) & γ(0) & ⋯ & γ(i-1) \\
      ⋮ \\
      γ(i) & γ(i-1) & ⋯ & γ(0)
    \end{pmatrix}^{-1}
    \begin{pmatrix}
      γ(1) \\ γ(2) \\ ⋮ \\ γ(i+1)
    \end{pmatrix}
    = 
    \begin{pmatrix}
      β_{i1} \\ β_{i2} \\ ⋮ \\ β_{ii}
    \end{pmatrix}
  \end{equation}
  and take the last element of $β(i)$ as $α(i)$.

  For an $AR(p)$, $α(i) = 0$ for all $ ≥ 0$

  For an $MA(q)$, the PACF dies out slowly as $i → ∞$.
\end{description}

\begin{itemize}
\item You can trivially extend ACF and PACF to vector processes and
  the same results broadly hold.
\item We're not going to worry about that now, because for
  multivariate dynamics we care mostly about finding structure that
  has an economic interpretation.
\item Sample ACF and PACF can be used for modeling.
\end{itemize}

\section{Wold Decomposition: the generality of VAR, VMA, and VARMA
  processes}

Introduce some probabilistic notation: define $F_t$ to be the sigma-field
(information set) generated by $y_t$, $y_{t-1}$, $y_{t-2},…$

\begin{description}
\item[Filtration:]
  A sequence $\{F_t\}$ of sigma-fields is a \emph{filtration} if
  $F_{t-1} ⊂ F_t$ for all $t$.

  The tail sigma-field of a collection of sigma-fields $\{F_t;
  t=…,-2,-1,0,1,2,…\}$ is defined as $⋂_{n=-∞}^∞ ⋃_{t=n}^∞ F_t$

  For a filtration, we have $⋃_{t=n}^{n+k} F_t = F_{n+k}$ and
  $⋃_{t=n}^k ⋂_{t=n+1}^k F_t = ⋃_{t=n+1}^k F_t$, so the tail
  sigma-field is equivalent to
  \begin{align}
    ⋂_{n=-∞}^∞ \lim_{k → ∞} ⋃_{t=n}^{n+k} F_t 
    &= ⋂_{n=-∞}^∞ \lim_{k → ∞} F_{n+k} \\
    &= \lim_{l,k → ∞} ⋂_{n=-l}^∞ F_{n+k} \\
    &= \lim_{l,k → ∞} F_{-l} \\
    &= \lim_{l → -∞} F_l \\
    &= F_{-∞}
  \end{align}

\item[Deterministic process:]
  A process is deterministic if it is perfectly predictable by linear
  combinations; i.e. $v_t$ is $G_{-∞}$-measurable, with $G_{-∞} = \lim
  G_n$ and
  \[G_n = \{c + \sum_{t ≤ n} a_t X_t ∣ c, a_t ∈ R\}.\]
\item[Wold decomposition:]
Any zero-mean nondeterministic stationary process $\{y_t\}$ can be
expressed as the sum of an $MA(∞)$ process $u_t$ and a deterministic
process $v_t$, so
\[y_t = \sum_{j=0}^∞ θ_j z_{t-j} + v_t\]
with $z_{t-j}$ in $G_n$ and $v_t$ in $G_{-∞}$; $z_t$ is the residual
from projecting $y_t$ onto $G_{t-1}$, and $v_t = y_t - θ(L) z_t$

Now, if $θ(L)$ is invertible then we can write this as a causal VAR,
and if $θ(L)$ can be factored, we can write the process as a causal
VARMA.
\end{description}

%%% Local Variables: 
%%% mode: latex
%%% TeX-master: "../textbook"
%%% End: 
% Copyright © 2013, Gray Calhoun.  Permission is granted to copy,
% distribute and/or modify this document under the terms of the GNU
% Free Documentation License, Version 1.3 or any later version
% published by the Free Software Foundation; with no Invariant
% Sections, no Front-Cover Texts, and no Back-Cover Texts.  A copy of
% the license is included in the section entitled "GNU Free
% Documentation License."

\chapter{Lectures on estimation (VARMA, VAR)}

Textbooks:
\begin{itemize}
\item Chapters 5 of Hamilton
\item Brockwell and Davis
\end{itemize}

\section{Parameter estimation}

\begin{itemize}
\item Method of moments
\item Maximum Likelihood (and conditional MLE)
\end{itemize}

\subsection{Estimation of VAR parameters}

Have the model (again):
\[y_t = μ + ∑_{i=1}^p φ_i y_{t-i} + e_t\] $k$ equations, $k p + 1$
regression coefficients. Under assumptions of causality, we know that
$e_t$ is uncorrelated with $y_s$ for $s < t$, which implies that we
might be able to justify OLS:
\begin{itemize}
\item Let $z_t' = (1 y_{t-1}' ⋯ y_{t-p}')$
\item $Φ = [ μ φ_1 … φ_p ]'$
\item Then $y_t' = z_t'Φ' + e_t'$ and we can do equation-by-equation
  OLS, which is equivalent to
  \[\hat Φ' = (∑_{t=p+1}^T z_t z_t')^{-1} ∑_{t=p+1}^T z_t y_t'\]
\end{itemize}

This is the method of moments estimator as well as the conditional MLE
estimator if you assume that $e_t ∼ N(0, Σ)$ and
condition on $y_1$,…,$y_p$. To do MLE, observe that
\[y_t ∣ z_t ∼ N(Φ z_t, Σ).\]

The joint likelihood can be taken as the product of the conditional
likelihoods:
\[L(Φ, Σ; y_1,…,y_T) = f_T(y_T ∣ y_{T-1},…, y_1) f_{T-1}(y_{T-1} ∣
y_{T-2},…, y_1) ⋯ f_{p+1}(y_{p+1} ∣ y_p,…,y_1) f_p(y_p,…,y_1)\]

Assuming normality and correct specification,
\[f_t(y_t ∣ y_{t-1},…, y_1) = f_t(y_t ∣ z_t).\]

If you want, you can avoid conditioning on $y_p$,…,$y_1$ since the
joint dist of those observations is also normal and we've worked out
how to find the mean and variance.

\subsection{Estimation for VARMA models (MA is a special case)}

Two approaches (we won't go into much detail):

\begin{enumerate}
\item Assume $e_t$ is normal, then $y_1,…,y_T$ is jointly normal and
  you can maximize the likelihood w/rt $Φ$, $Θ$, and $Σ$
\end{enumerate}

\begin{itemize}
\item Amounts to invoking invertibility to derive $f_T(y_t ∣
  y_{t-1},…,y_1)$ efficiently for each $t$ w/out using strict VAR
  structure, then building up the unconditional likelihood from that.
\end{itemize}

\begin{enumerate}
\item Just like for VAR and VMA processes, we can derive equations
  that define the autocovariances at lags $0,…,p+q$ in terms of $Φ$
  and $Θ$; solving for $Φ$ and $Θ$ and plugging in the sample
  autocovariances give the ``Yule-Walker'' estimates.
\end{enumerate}

\begin{itemize}
\item For VARs, this is OLS
\item When there's an MA component, this can be inefficient.
\end{itemize}

\section{Asymptotic theory for time-series}

Look at OLS estimator for VAR(p). As always, two components:
\[\hat Φ' = Φ' + (1/T ∑_{t=p+1}^T z_t z_t')^{-1} 1/T
∑_{t=p+1}^T z_t e_t'\] so, for consistency need the sum of $z$'s to be
$O_p(1)$; second sum to be $o_p(1)$. For asymptotic distribution
(normality) need second summation to obey a CLT.

\begin{itemize}
\item Obviously, if we want to estimate the first or second moments of
  a time-series process, we might want to use the sample moments.
\item Of interest on their own
\item Necessary for consistency and asymptotic normality of MoM
  estimators
\item Just look at univariate for simplicity.
\end{itemize}

\begin{description}
\item[Filtration:]
  A sequence of sigma-fields $\{G_t\}$ is a \emph{filtration} if $G_t
  ⊂ G_{t+1}$ for all $t$
\item[Martingale difference sequence:]
  The sequence of rvs $\{e_t\}$ is an mds w/rt the filtration $\{G_t\}$ if
  $E(e_t ∣ G_{t-1}) = 0$.

  Typically $G_t = σ(e_t, e_{t-1},…)$

  Comes from the definition of a martingale: $y_t$ is a martingale
  w/rt the filtration if $E(y_t ∣ G_{t-1}) = y_{t-1}$. Then $e_t =
  Δy_t$ is an mds.
\end{description}

\begin{itemize}
\item Now, suppose that $e_t$ has finite variance and is an mds w/rt
  $G_t = σ(y_t, y_{t-1},…)$, then we know that the second sum has mean
  zero. Also
  \begin{align*}
    E (1/T ∑ z_t e_t)(1/T ∑ e_s z_s')
    &= 2/T^2 ∑_{s ≤ t} E(e_s e_t z_t z_s' ∣ G_{t-1}) \\
    &= 1/T^2 ∑_t E E( e_t^2 ∣ G_{t-1}) z_t z_t' + 2/T^2 ∑_{s < t} E E(
    e_t ∣ G_{t-1}) e_s z_t z_s' \\
    &= σ^2 1/T^2 ∑_t E z_t z_t'
  \end{align*}

  This converges to zero since $y_t$ has finite variance and Φ-hat
  converges in MSE to $Φ$.
\item Note that if $e_t$ is MDS, so is $z_t e_t$
\end{itemize}

\subsection{Standard MDS CLT (McLeish 1974; Hall and Heyde 1980)}

Suppose that $Z_t$ is a (univariate) MDS and define
\begin{itemize}
\item $U_n^2 = ∑_1^n Z_t^2$,
\item $s_n^2 = E U_n^2$.
\end{itemize}
If
\begin{itemize}
\item $U_n^2 / s_n^2 → 1$ i.p. and
\item $max_{1 ≤ t ≤ n} |Z_n / s_n| → 0$ i.p.
\end{itemize}
then
\[1/s_n ∑_1^n Z_t →^d N(0, 1)\]
or, equivalently,
\[1/U_n ∑ Z_t →^d N(0, 1)\]

For multivariate, let $Ω_n = ∑ E Z_t Z_t'$ and we have $Ω_n^{-1/2} ∑
Z_t → N(0, I)$ under essentially the same conditions.

Apply this to second summation in OLS coefficients and (assuming the
conditions hold) we can see that
\[n^{-1/2} ∑ z_t e_t = ( Ω_n / n )^{1/2} Ω_n^{-1/2} ∑ z_t e_t
→^d N(0,Ω) \]

where $Ω_n = ∑ E( e_t^2 z_t z_t' )$ and $Ω = \lim Ω_n / n$. Under
homoskedasticity, this can be simplified.

If $1/n ∑ z_t z_t' → V$ i.p., then $\lim Ω_n = σ^2 V$ under
homoskedasticity and we have
\[\sqrt{n} ( \hat φ - φ ) →^d N( 0, σ^2 V^{-1} ).\]

Under heteroskedasticity,
\[\sqrt{n} (\hat φ - φ ) →^d N( 0, V^{-1} Ω V^{-1} )\]

\textbf{For the asymptotic distribution of the VAR(p)} obviously it's
joint normal, so the only tricky thing is accounting for all of the
correlations. Rewrite it as
\[
sqrt(n) (\hat Φ' - Φ')
= (1/n ∑_{t=p+1}^n z_t z_t')^{-1} 1/sqrt(n) ∑_{t=p+1}^n z_t e_t'
\]
so
\[\sqrt{n} ( vec(\hat Φ' ) - vec( Φ' ) ) = vec( … )\]

and use the fact that, for conformable matrices $A$, $B$, and $C$,
\[vec( A B C ) = ( C ⊗ A ) vec( B )\]
where $⊗$ is the kronecker product and
\[
C ⊗ A = 
\begin{pmatrix}
  c_{11}A & ⋯ & c_{1n}A \\
  ⋮
  c_{n1}A & ⋯ & c_{nn}A 
\end{pmatrix}
\]
$A = ( 1/n ∑ z_t z_t' )^{-1}$, $B = 1/sqrt{n} ∑ z_t e_t'$, $C = I$ so
$C ⊗ A = diag( A,…, A )$

if $V = plim 1/n ∑ z_t z_t' Ω = plim 1/n ∑ vec( z_t e_t' ) vec( z_t
e_t' )'$, then $\sqrt{n} vec(\hat Φ' - Φ' ) → N( 0, diag(
V^{-1},…, V^{-1} ) Ω diag(V^{-1},…,V^{-1} )$

Note that the block diagonal structure means that if we're interested
in the joint dist. of the coefficients for just a single equation,
it's not affected by those from other equations (only holds when all
of the equations have the same regressors…).

\subsection{Result (in summary)}

Suppose that $y_t$ is a VAR(p),
\begin{itemize}
\item $1/n ∑ z_t z_t' → V$ i.p.
\item $1/n ∑ z_t z_t' e_t^2 → Ω$ i.p.
\item $e_t$ is an mds with finite $4 + δ$ moments ($δ > 0$)
\end{itemize}
Then $\sqrt{T} ( \hat Φ' - Φ' )$ is asymptotically normal (with the
variance we derived).

\subsection{Laws of large numbers}
We still need to talk about the square terms:
\[1/n ∑ z_t z_t'\]
and
\[1/n ∑ vec( z_t e_t' ) vec( z_t e_t' )'\]
\begin{description}
\item[LLN for martingale difference sequences:]
If $z_t$ is uniformly integrable (i.e. a slightly weaker condition than
assuming it has $1 + δ$ moments, $δ > 0$) and MDS, then
\[1/n ∑ z_t →^{L_1} 0\]
\end{description}

MDS for second moments is not usually an implication of Economic
Theory; is not usually a feature of a correctly specified model of the
mean; and is often false in Economic data. So it's worth mentioning
another result:

\begin{description}
\item[LLN for stationary and ergodic sequences:]

  If $\{z_t\}$ is a stationary and ergodic sequence with finite mean
  $μ$, then
  \[1/n ∑ z_t → μ\] (in $L_1$).

  Ergodic: a stationary sequence $\{x_t\}$ is \emph{ergodic} if, for
  any bounded functions $f$ and $g$, as $n → ∞$,
  \[E f( x_t, …, x_{t + k}) g(x_{t + n}, …, x_{ t + n + l} ) → E f(
  x_t, …, x_{ t + k } ) E g(x_{ t + n }, …, x_{ t + n + l} )\]

  \begin{itemize}
  \item remember, you can prove independence by proving this
    factorization for all bounded $f$ and $g$, so this is an ``asymptotic
    independence'' condition;
  \item Sometimes see conditions on how fast the convergence to zero
    happens
  \item Nothing special about stationarity; you'll see other weak
    dependence conditions based on similar ideas (i.e. mixing
    conditions).
  \end{itemize}
\end{description}

\subsection{CLT with serial correlation}

Have a CLT for MDS. But what if we have an MA process?
\[y_t = ∑_{j=0}^∞ θ_j e_{t-j} = θ(L) e_t\]

\begin{itemize}
\item Just focus on the average: $̱\bar y$
\item Obviously, if $θ_j$ decays fast enough, LLN holds.
\item Look at CLT:
  \begin{align*}
    T^{1/2} \bar y
    &= 1/T^{1/2} ∑_{t=1}^T ∑_{j=0}^∞ θ_j e_{t-j} \\
    &= \sqrt{1/T} ( θ_0 e_T +
    ( θ_0 + θ_1 ) e_{T-1} +
    ( θ_0 + θ_1 + θ_2 )  e_{T-2} + ...
    ( θ_0 + ... + θ_{T-1} ) e_1 + ... ) \\
    &= \sqrt{1/T} ( ( θ(1) - ∑_1^∞ θ_j ) e_T +
    ( θ(1) - ∑_2^∞ θ_j ) e_{T-1} + ...
    ( θ(1) - ∑_T^∞ θ_j ) e_1 + ... ) \\
    &= \sqrt{1/T} θ(1) ∑_{t=1}^T e_t + \sqrt{1/T} θ^*(L) e_t
  \end{align*}
  where $θ_j^* = - ∑_{i=j+1}^∞ θ_i$.  Now $θ(1)$ has the bulk of the
  dependence, so the first term obeys a CLT and the second can
  converge to zero i.p.
\end{itemize}

Result: Suppose that $y_t = μ + θ(L) e_t$ where
\begin{itemize}
\item $∑_{s=0}^∞ s |θ_s| < ∞$
\item $1/\sqrt{T} ∑_t e_t → N(0, σ^2)$ in distribution (i.e. any CLT
  holds)
\end{itemize}
Then $\sqrt{T} (\bar y - μ ) → N( 0, σ^2 θ(1)^2 )$ in distribution

\section{Last bits}

Forecasting from VAR:
\begin{itemize}
\item 1-step ahead $\hat y_{T+1} = \hat φ(L) y_T$
\item $k$-step ahead $\hat y_{T+k} = \hat φ(L) y_{T+k-1} = … = ( \hat
  φ(L) )^k y_T$
\end{itemize}
Also see forecasting by just regressing $y_{t+k}$ on $y_t,…,y_{t-p}$

What to do with these approximations?
\begin{itemize}
\item Test as normal
\item Test for Granger causality/exogeneity
  \begin{itemize}
  \item split up our variables into
    \begin{align*}
      y_{1t} &= μ_1 + A_1 x_{1t} + A_2 x_{2t} + e_t \\
      y_{2t} &= μ_2 + B_1 x_{1t} + B_2 x_{2t} + e_t
    \end{align*}
    where the $x_{it}$ contains lags of $y_{it}$.
  \item $y_2$ does not granger-cause $y_1$ if lags of $y_2$ do not
    help predict $y_1$ after accounting for lags of $y_1$ (also means
    that $y_1$ is exogenous w/rt $y_2$).
  \item Means that $A_2 = 0$.
  \item This is something we can test.
  \end{itemize}
\end{itemize}

\subsection{Lag length selection}

\begin{itemize}
\item $AIC_p = \log | Σ_u(p)| + 2 p m^2 / T$
\item $BIC_p = \log | Σ_u(p)| + \log(T) 2 p m^2 / T$
\item $Σ_u(q)$ is the vcv of $y_t - b_1 y_{t-1} - … - y_{t-p}$
\item Choose to minimize the criterion
\item Last term works as a penalty
\end{itemize}

%%% Local Variables: 
%%% mode: latex
%%% TeX-master: "../textbook"
%%% End: 
% Copyright © 2013, Gray Calhoun.  Permission is granted to copy,
% distribute and/or modify this document under the terms of the GNU
% Free Documentation License, Version 1.3 or any later version
% published by the Free Software Foundation; with no Invariant
% Sections, no Front-Cover Texts, and no Back-Cover Texts.
% 
% You should have received a copy of this license along with this
% document.  If not, you can find one at
% <http://www.gnu.org/copyleft/fdl.html>

\part*{Structural VARs}
\addcontentsline{toc}{part}{Structural VARs}

Textbook material:
\begin{itemize}
\item Greene 6th edition, Chapter 19
\item Greene 7th edition, Chapter 20
\end{itemize}
Review articles:
\begin{itemize}
\item Kilian's recent (2011?) handbook chapter
\item Stock and Watson (2001) \emph{JEP}. ``Vector Autoregressions''
\item Watson (1994) \emph{Handbook}. ``Vector autoregressions and
  cointegration'' (but it's pretty old at this point).
\end{itemize}
Important background:
\begin{itemize}
\item Sims (1980) Macroeconomics and Reality
\end{itemize}
Other books that students may want to consult:
\begin{itemize}
\item Jim Hamilton (1994), \emph{Time series analysis}.
\end{itemize}

\section{Quick: forecast from VAR(p)}

\begin{itemize}
\item Iterated forecasts
\item Direct forecasts
\end{itemize}

\subsection{Uncertainty}

\begin{itemize}
\item Want confidence intervals associated with these IRFs (draw)
\item $δ$-method (example for AR(1)): 
  \[y_t = ρ y_{t-1} + e_t,\]
  then we know that 
  \[  y_{t+k} - y_{t-1} = ρ^k Δe_t;\] 
  so we'd like to make a confidence region for
\end{itemize}
\[\sqrt{T} (\hat ρ^1 - ρ^1,…,\hat ρ^k - ρ^k)\]
for any $j$, $\sqrt{T} (\hat ρ^j - ρ^j) → N(0, (j ρ^{j-1}) 2 σ²)$,
which breaks if $ρ = 0$!

\begin{itemize}
\item For matrices, spots where it breaks are less obvious and depend
  on the derivative of the AR coefficients.
\item Bootstrap also used (but suffers from some of the same problems
  as $δ$-method)
\item grid bootstrap (Mikusheva 2012)
\item mostly get pointwise CIs
\end{itemize}

(Add material on Bayesian CIs too)

\section{Basic ideas for IRF}

AR and MA models that we discussed earlier are \emph{descriptive}
models, but don't necessarily have any economic content
\[φ(L) y_t = θ(L) e_t\]
describes the second moments of $y_t$; the usual concerns about
simultaneity apply here (and even more forcefully than usual).

\begin{itemize}
\item Interested in dynamics
\item Want to know, what happens to $y_{t+1}, y_{t+2},…, y_{t+h},…$
  given an exogenous change to $y_{1,t}$?
\item What is an exogenous change to $y_{1,t}$? Can it be isolated
  from changes to $y_{2,t}$, $y_{3,t}$, etc?
\item Monetary shock
\item Technology shock
\item Financial shock (credit…)
\item Oil price shock
\item UNANTICIPATED
\end{itemize}

Look at $MA(∞)$ representation (assuming stationarity, etc) zero mean
for simplicity \[y_t = C(L) e_t\] where $C(L) = φ(L)^{-1} θ(L) e_t$

we want to transform/rotate $e_t$ so that we have ``structural''
shocks $u_t$,
\[u_t = V e_t\] typically $u_t$ is normalized so that
\[var( u_t ) = I\] (uncorrelation is kind of the definition of a macro
shock) and we have
\[y_t = C(L) V^{-1} u_t\ (rewrite\ as)\ B(L) u_t\]

So then
\[y_t = B_0 u_t + B_1 u_{t-1} + B_2 u_{t-2} + ⋯\]
and the effect of
\begin{itemize}
\item $u_{1t}$ shock to $y_t$ is $B_0 (1 0 0 … 0)$
\item $u_{1t}$ shock to $y_{t+1}$ is $B_1 (1 0 0 … 0)'$
\item etc.
\item $u_{1t}$ shock to $y_{t+k}$ is $B_k (1 0 0 … 0)'$
\end{itemize}
notice that $V$ is not identified and so the shocks are not identified:
if $W$ is any orthonormal matrix, so that $W W' = I$, then $W u_t$ has
variance $I$ as well, and
\[C(L) e_t = C(L) V^{-1} u_t = C(L) (VW)^{-1} ( W u_t )\]

\subsection{VAR representation}

(mention rank conditions)

\[B(L)^{-1} = A(L);\]
\begin{itemize}
\item structural form: $A_0 y_t = A_1 y_{t-1} + ⋯ + A_p y_{t-p} + u_t$
\item reduced form: $y_t = A_0^{-1} A_1 y_{t-1} + ⋯ + A_0^{-1} A_p
  y_{t-p} + A_0^{-1} u_t$
\end{itemize}
in this notation, the effect of
\begin{itemize}
\item $u_{1t}$ shock to $y_t$ is $A_0^{-1} (1 0 … 0)'$
\item $u_{1t}$ shock to $y_{t+1}$ is $A_0^{-1} A_1 × shocked\ y_t =
  A_0^{-1} A_1 A_0^{-1} (1 0 … 0)'$
\item $u_{1t}$ shock to $y_{t+2} = A_0^{-1} A_1 × shocked\ y_{t+1} +
  A_0^{-1} A_2 × shocked\ y_t = (A_0^{-1} A_1) × (A_0^{-1} A_1)
  A_0^{-1} (1 0 ⋯ 0)'$
  \begin{itemize}
  \item $(A_0^{-1} A_2) * A_0^{-1} (1 0 ⋯ 0)'$
  \end{itemize}
\end{itemize}
etc (these formulas are really easy if you use the canonical form of
the VAR)

\begin{itemize}
\item With OLS, we can consistently estimate $A_0^{-1} A_1$, $A_0^{-1}
  A_2$, etc and $A_0^{-1} A_0^{-1\prime}$ (i.e. the variance-covariance
  matrix of the reduced form error).
\item BUT, we actually need to know \emph{all} of the elements of
  $A_0$ or $A_0^{-1}$, and $A_0^{-1} A_0^{-1\prime}$ is positive definite
  and so has only $n (n-1) / 2$ unique elements, not the $n^2$.
\end{itemize}

\section{Bayesian IRFs}


\begin{itemize}
\item specify prior for AR coefficients and VCV
\item Draw candidate values of the AR coefficients from the posterior
\item For each one, calculate the IRF as before
\item Summarize with mean, mode, or credible set
\item It is \emph{very unlikely} that these are uniformly valid
  credible sets if you use the uniform prior
\end{itemize}

\section{Identification}

\begin{itemize}
\item Short run identification: Constrain $A_0$ or $A_0^{-1}$, not
  $(A_0^{-1} A_i)$.
\item that would be the typical simultaneous equations approach, IV,
  etc
\item Long-run identification (we'll discuss in a bit) -- based on the
  idea that some shocks (productivity shocks for example) permanently
  affect the level of some variables (per-capita output, for example)
\item Partial identification
\end{itemize}

\subsection{Short-run identification}

Suppose we know that
\begin{itemize}
\item $y_{1t}$ ignores $u_{2t}$ through $u_{nt}$
\item $y_{2t}$ ignores $u_{3t}$ through $u_{nt}$
\item dot dot dot
\item $y_{n-1,t}$ ignores $u_{nt}$
\item $y_{n,t}$ responds to all of the shocks
\end{itemize}
Then you know that $A_0^{-1}$ is lower triangular and so it's
identified.

\begin{itemize}
\item Sounds sort of ridiculous, but this was Sims's (1980) first
  identification approach;
\item You could imagine that the Federal Reserve sees what's happened
  in the economy before acting, so it would be at the bottom; $u_{nt}$
  would represent a monetary policy shock (the equation for $y_{nt}$
  would be something like a Taylor rule and $y_{nt}$ needs to be an
  interest rate).
\item ie imagine inflation, unemployment, and interest rates
\item probably want to leave the other coefficients of the interest
  rate equation unspecified so that you're allowing it to be forward
  looking (or include additional forward looking variables)
\item could specify 
  \begin{equation*}
    R_t = R^* + coef * (inflation_t - inflation^*) + coef * (unemp_t - unemp^*) + coefs *
    R_{t-1} + ⋯ + e_{Rt}
  \end{equation*}
\item but now this equation needs to be right (if it is, another
  source of identification)
\item This has the ``advantage'' of being especially easy to calculate
  (comes from a Choleski decomposition of the sample covariance
  matrix).
\item This is kind of old-fashioned; you can't usually identify every
  shock this way;
\end{itemize}

\subsection{semi-structural}
\[y_t = (Δ gdp_t,\ inflation_t,\ fed\ funds_t)\]
``identify'' with recursive structure: 
\begin{itemize}
\item $Δ gdp_t$ only responds to
``first shock''
\item $inflation_t$ only responds to ``first shock'' and
``second shock''
\item $fed\ funds_t$ is set by monetary policy and responds
to everything
\item important thing is that it happens last
\item identify ``monetary policy shock'' as deviations from
  ``endogenous policy''
\item don't identify any other shocks (recursive structure for first
  two is really just a normalization device).
\end{itemize}

\begin{itemize}
\item This could be more convincing with higher-frequency data (ie
  kalman filter/ state space models) or better policy knowledge
\item An argument that prices respond more quickly than quantities
  seems plausible overall (remembering that the VAR is supposed to be
  taking care of the endogenous component of both)
\end{itemize}

\subsection{Kilian (2010) has other examples}

\begin{itemize}
\item Any economic argument to pin down particular values of
  $A_0^{-1}$ or $A_0$ can do the same trick.
\end{itemize}

\subsection{Long-run identification}

Introduced in Blanchard and Quah (1989 AER)

\begin{itemize}
\item Their model: bivariate VAR ($Δ gdp_t$, $u_t$)
\item $Δ gdp_t$ is I(0)
\item Assume for argument's sake that $gdp_0 = 0$
\item Assumption: want to identify the shocks that have a long-run
  effect on $gdp_t$ (they argue that these have interpretation as a
  supply shock)
\item Start with structural MA representation:
\item $v_t$ consists of the policy shocks; first element is ``supply''
  shock
  \[( Δ gdp_t, u_t ) = C(L) e_t = D(L) v_t\]
  Simple rewrite:
  \[ ∑_{s=1}^t (Δ gdp_s, u_s) = D(L) ∑_s v_s \]
\item Apply the ``beveridge-nelson decomposition'' (note that $D(1) =
  D_0 + D_1 + …$
  \begin{align*}
    ∑_{s=1}^t D(L) v_s
    &= ∑_{s=1}^t ∑_{j=0}^∞ D_j v_{s-j} \\
    &= ∑_{j=0}^∞ v_{t-j} ∑_{s=0}^j D_s \\
    &= (D(1) - ∑_{j=1}^∞ D_j) v_t + (D(1) - ∑_{j=2}^∞ D_j) v_{t-1} + ⋯ \\
    &= D(1) ∑_{s=1}^t v_s + D^*(L) v_t
  \end{align*}
  where $D_k^* = - ∑_{j=k+1}^∞ D_j$ and
  $D(1) = \begin{pmatrix} 
    D_{11}(1) & D_{12}(1) \\ D_{21}(1) & D_{22}(1)
  \end{pmatrix}$
\item This gets us $(gdp_t, ∑_{s=1}^t u_s) = sum = D(1) ∑_{s=1}^t v_s
  + D(L) v_t$
\item second part are purely transitory shocks
\item $D(1)$ part are the ``permanent shocks''
\item model makes no claim about permanent
  shock on summed unemployment
\item does make claim about shock on
  $gdp_t$
\item second element of $v_t$ can't have permanent effect, so
  $D_{12}(1)$ must be zero!
  \[vcv(v_s) = I vcv(C(1) e_t) = vcv(D(1) v_t),\]
  so
  \begin{itemize}
  \item $C(1) Σ C(1)`= D(1) D(1)'$ (and LHS can be estimated
    consistently);
  \item here $D(1)$ can be estimated as the choleski decomposition of
    estimated $C(1) Σ C(1)'$ (and know that, if we have the reduced
    form VAR model $Φ(L) y_t = e_t$ then $Φ(L)^{-1} = C(L)$ (and this
    holds for $L = 1$)
  \item So then $v_t = \hat D(1)^{-1} \hat C(1) e_t$ and $A_0^{-1} =
    \hat D(1)^{-1} \hat C(1)$
  \end{itemize}
\end{itemize}

\subsection{Partial identification through sign restrictions.}

The basic idea behind partial identification is that we may not have
enough information to pin down a value precisely, but we might still
be able to derive economically interesting restrictions.  For example,
if we define a set of \emph{potential} ``loose'' monetary policy
shocks to be any innovation that lowers the Federal Funds rate, and
\emph{any} shock that meets that minor condition leads to a rise in
GDP growth, we can conclude that monetary policy shocks cause GDP
growth to rise.

\begin{itemize}
\item The key difference between this approach and the previous
  approaches is that, in this hypothetical example, we don't need to
  know which of the innovations actually line up with monetary policy
  shocks, since we know that (hypothetically) GDP growth has a
  nonnegative response to all of them.
\item Implementation without worrying about inference is relatively
  easy:
  \begin{itemize}
  \item Estimate the reduced-form VAR,
    \[ y_t = Φ₀ + ∑_{j=1}^p Φ_j y_{t-j} + e_t \]
    and the variance-covariance matrix of $e_t$ as usual.
  \item Cholesky-decompose $\Σh$, so $\Σh = PP'$ with $P$ lower
    triangular.
  \item For any orthonormal $D$, $\Σh = PP'$ implies that $\Σh =
    (PD)(PD)'$, so we're going to proceed as before, but letting
    $A₀^{-1} = PD$ and letting $D$ range over the whole set of
    orthonormal matrices.  ``Range'' means that we'll simulate a bunch of
    orthonormal matrices.
  
    One way to let $D$ range over these matrices is to draw a $k×k$
    matrix of standard normal random variables, call it $L$, and take
    the QR decomposition of $L$, giving $L=QR$ where $Q$ is
    orthonormal.  Then let $D=Q$.
  \item For each draw of $A₀=(PD)^{-1}$, calculate the IRFs.  If they
    satisfy some economically-motivated constraints, keep that draw of
    $A₀$ (in the brief example above, the constraint would be that the
    Federal Funds rate has an immediate positive response).  Otherwise
    discard it.  Either way, draw many many more candidate values of
    $A₀$.
  \item The set of unrejected $A₀$ defines a set of potential IRFs for
    the economic shock of interest.  Note that there's no way to say
    that one of these IRFs is ``more plausible'' than any others,
    since they all correspond to the exact same value of the
    likelihiood.
  \end{itemize}
\item Bayesian estimation works very nicely with the algorithm
  described above, and makes it computationally easy to account for
  uncertainty in the estimators of $Φ$ and $Σ$.  But we want to be
  careful to not accidentally turn the method for generating candidate
  values of $A₀$ into a prior over different values of $A₀$, which
  will happen if we treat uncertainty over $A₀$ the same as
  uncertainty over the values of $Φ$ and $Σ$.  (Need to add references)
\item I need to add notes and references on handling estimation
  uncertainty for classical estimation and for Bayesian estimation
  when we don't want to impose a prior on $A₀$.
\end{itemize}

%%% Local Variables: 
%%% mode: latex
%%% TeX-master: "../textbook"
%%% End: 
% Copyright © 2013, Gray Calhoun.  Permission is granted to copy,
% distribute and/or modify this document under the terms of the GNU Free
% Documentation License, Version 1.3 or any later version published by
% the Free Software Foundation; with no Invariant Sections, no
% Front-Cover Texts, and no Back-Cover Texts.  A copy of the license is
% included in the section entitled "GNU Free Documentation License."

\part{Stochastic integration}

\begin{itemize}
\item
  $W(λ)$ is a \emph{Weiner process} or \emph{Brownian Motion} if it is
  \begin{itemize}
  \item continuous and mean zero
  \item $W(t) - W(s) \sim N(0, t-s)$ for any $t$ and $s$
  \item Non-overlapping intervals are independent
  \end{itemize}
\item Draw diagram of sample paths w/ $\sqrt T$ envelope
\item Obviously, if $e_t$ is an MDS with variance 1, we have
  \[1/\sqrt{T} \sum_{t=1}^{[λT]} e_t →^d W(λ)\]
  under reasonable assumptions; want to be able to write this as an
  integral $\int_0^λ dW$

  \begin{itemize}
  \item
    $\lim 1/T \sum_{t=1}^{[λT]}$ maps to $\int_0^λ dt$
  \item
    $\lim e_t \sqrt{T}$ must map to $dW/dt$

    \begin{itemize}
    \item $e_t/\sqrt{T} ≈ dW$
    \item i.e. $dW^2 ≈ dt$
    \item You can also see this from writing out
      \[dW/dt = \lim_{h → 0} (W(t+h) - W(t)) / h\]
    \end{itemize}
  \item This is just intuition; proving is a little more difficult.
  \item Implies that $W(λ)$ is not differentiable a.e.
  \end{itemize}
\item We only need a few basic results; there are entire classes you
  can take on working with Ito integrals and SDEs
\end{itemize}

\section{Application \& use}

\begin{itemize}
\item Continuous mapping theorem: if $f$ is a continuous functional on
  $[0,1]$ then \[f(1/\sqrt{T} \sum_{t=1}^{[λT]} e_t) →^d f(W(λ))\]
  and, generally, if $g_t →^d g$ where $g$ is a random process on
  $[0,1]$ then $f(g_t) →^d f(g)$
\item Functional delta-method is similar
\item Also can use integration results
\end{itemize}

\section{Unit roots in regression}

\subsection{Spurious regression}

Suppose that we have a multivariate unit root process
\[y_t = y_{t-1} + e_t\] and we run the regression
\[y_{1t} = β y_{2,t-1} + u_t\]

Define

\begin{itemize}
\item $v_1 = (1, 0)'$
\item $v_2 = (0, 1)'$
\end{itemize}

The OLS coefficients can be written as
\begin{equation}
  \begin{split}
    \hat β &= \Big(\sum_{t=2}^T v_2' y_{t-1} y_{t-1}' v_2\Big)^{-1}
    \sum_{t=2}^T v_2' y_{t-1} y_t' v_1 \\
    &= \Big(v_2' \sum_{t=2}^T y_{t-1} y_{t-1}' v_2 \Big)^{-1} \Big(
    1/T^2 v_2' \sum_{t=2}^T y_{t-1} y_{t-1}' v_1
    + 1/T^2 v_2' \sum_{t=2}^T y_{t-1} e_t v_1 \Big) \\
    &→^d \Big(v_2' Σ^{1/2} \int_0^1 W(s) W(s)' ds Σ^{1/2}
    v_2\Big)^{-1} v_2' Σ^{1/2} \int_0^1 W(s) W(s)' ds Σ^{1/2} v_1
\end{split}
\end{equation}
This is interesting because it means that $\hat β$ is not consistent
(i.e.~doesn't converge to $β$). You can also show that $t$-stats don't
work (the statistic diverges so the test rejects with probability 1 in
the limit)

Relevant texts:
\begin{itemize}
\item Granger and Newbold (1974)
\item Phillips (1986)
\end{itemize}

\section{Regress I(1) on I(0)}

Take the last example, but suppose that we include a covariance
stationary I(0) regressor $x_{t-1}$ (but this is going to work for any
collection of I(0) regressors as well),
\[y_t = β_0 x_{t-1} + β_1 + β_2 y_{t-1} + u_t\] and say we want to get
limiting distributions for OLS. Without loss of generality, assume that
$E x_t = 0$; otherwise rewrite the equation as
\[y_t = β_0 (x_{t-1} - E x_{t-1}) + (β_1 + β_0 E x_{t-1}) + β_2 y_{t-1} + u_t\]
Also assume that $var e_t$ is 1 to keep the notation as simple as
possible.

How to get asymp dist of $\hat β - β$? Key thing is that the different
elements are going to converge at different rates; let
\[Λ = diag(\sqrt{T}, \sqrt{T}, T)\] so
\begin{equation}
\begin{split}
  Λ (\hat β - β) &=  \left( Λ^{-1} \sum_{t=2}^T
    \begin{pmatrix}
      x_{t-1}^2       & x_{t-1}   & x_{t-1} y_{t-1} \\
      x_{t-1}         & 1         & y_{t-1} \\ 
      x_{t-1} y_{t-1} & y_{t-1}   & y_{t-1}^2
    \end{pmatrix} Λ^{-1} \right)^{-1}
  Λ^{-1} \sum_{t=2}^T 
  \begin{pmatrix}
    x_{t-1} e_t \\ e_t \\ y_{t-1} e_t 
  \end{pmatrix}\\
  & \to^d
  \begin{pmatrix}
    E x_t^2 & 0             & 0 \\
    0       & 1             & \int_0^1 W(s) ds \\
    0       & \int_0^1 W(s) ds & \int_0^1 W(s)^2 ds
  \end{pmatrix}^{-1}
  \begin{pmatrix} (E x_t^2)^{1/2} W(1) \\ W(1) \\ \int_0^1 W(s) dW(s) \end{pmatrix}
\end{split}
\end{equation}
Where the 0 terms in the X'X matrix come from
\[T^{-1} \sum_{t=2}^T x_{t-1} →^p 0\] from the LLN and
\[T^{-1} \sum_{t=2}^T x_{t-1} y_{t-1} →^d \int_0^1 W_y(s) dW_x(s)\] so
\[T^{-3/2} \sum_{t=2}^T x_{t-1} y_{t-1} →^p 0.\]

Since the $X'X$ component is block diagonal, the asymptotic distribution
of the first element of $\hat β$ is $(E x_t^2)^{-1} W(1)$, making it
consistent and asymptotically normal at the usual rate. The coefficient
on the I(1) term and on the constant are both not Normal.

Note that I got a little too cute in class and forgot that the constant
term is \emph{not} I(0); note that $T^{-1} \sum_t 1$ obeys a ``LLN'' and
converges to 1, but $T^{-½} \sum_t 1$ definitely does not obey a ``CLT''.
So the estimator on $x_{t-1}$ is asymptotically normal, but the
estimator of the intercept is not.

\textbf{Additional lag structure}

Suppose now you run the regression
\[y_t = β_0 + β_1 y_{t-1} + β_2 y_{t-2} + u_t\]

\begin{itemize}
\item know that OLS estimator of $β_0$ is normal and correctly centered
\item we can rewrite the relationship as
  \[y_t = β_0 + β_1 Δy_{t-1} + (β_2 + β_1) y_{t-2} + u_t\] and
  estimating $β_1$ in this equation will give

  \begin{itemize}
  \item A numerically identical estimate as in the previous equation
  \item A consistent and asymptotically normal estimator of $β_1$
  \item Note that our estimate of $β_1 + β_2$ will have an awkward
    distribution
  \item So the OLS estimate of $β_1$ in the original regression is
    consistent and asymptotically normal
  \end{itemize}
\item Similarly, we can show that the OLS estimate of $β_2$ in the
  original regression is consistent and asymptotically normal.
\item Note that the estimate of $β$ is not jointly normal, since
  $β_1+β_2$ has a non-normal distribution.
\item This is true whenever you can rewrite the expressions so that
  coefficients appear on I(0) components and has implications for
  cointegration.
\end{itemize}

Relevant texts:
\begin{itemize}
\item Sims, Stock, and Watson (1990)
\end{itemize}

\section{Regress I(0) on I(1)}
To be added\ldots

%%% Local Variables: 
%%% mode: latex
%%% TeX-master: "../textbook"
%%% End: 
% Copyright © 2013, Gray Calhoun.  Permission is granted to copy,
% distribute and/or modify this document under the terms of the GNU
% Free Documentation License, Version 1.3 or any later version
% published by the Free Software Foundation; with no Invariant
% Sections, no Front-Cover Texts, and no Back-Cover Texts.
% 
% You should have received a copy of this license along with this
% document.  If not, you can find one at
% <http://www.gnu.org/copyleft/fdl.html>

\part*{Cointegration}
\addcontentsline{toc}{part}{Cointegration}

Look at VAR(1): $y_t = a_0 + A y_{t-1} + e_t$; remember that if we
want something like stationarity, we need to look at the roots of the
polynomial $\det(I - A z) = 0$; nonstationary if there are roots on
the unit circle (ie $|z| = 1$);

possibilities:
\begin{itemize}
\item $n$ roots ($n$ is the number of equations) on the unit circle
\item less than $n$, but positive
\item no roots on unit circle
\end{itemize}

We haven't formally defined what it means for a process to be $I(0)$
yet, but it captures what we usually mean when we say that a series is
stationary.  If a series is stationary, we typically expect that it
obeys a CLT.  But it is straightforward to write down stationary
series that don't (for example, let $u_t$ be an i.i.d. sequence; then
$Δu_t$ is stationary but obviously does not obey a CLT).

So we say that the series $u_t$ is $I(0)$ if it has the MA
representation $u_t = Φ(L) ε_t$ where $ε_t$ is white noise, $Φ(1) ≠
0$, and $∑_{j=0}^∞ | Φ_j |$ is finite (element-by-element in the case
of multivariate processes).

Now we extend the definition to $I(d)$ recursively: for $d > 0$, $u_t$
is $I(d)$ if $Δu_t$ is $I(d-1)$ and is $I(-d)$ if $∑_{j=-∞}^∞ u_t$ is
$I(1-d)$.

\paragraph{Cointegration setup}
Let $Π = A - I$ and rewrite the VAR as
\[Δ y_t = a_0 + Π y_{t-1} + e_t\]

$Π$ may not have full rank, but if it has rank $r$, we can always write $Π =
αβ'$ where $α$ and $β$ both are $k × r$ with full rank.
Then \[Δy_t = a_0 + αβ'y_{t-1} + e_t\]
(essentially, this is the ``Granger representation theorem'')

\paragraph{Why must it be $β'y_t$ that's $I(0)$?} The intuition's
easiest to see when $r=1$.  In that case, $α$ becomes just a vector of
constants, so it can't possibly remove the stochastic trend.  

why do this: 
\begin{itemize}
\item LHS is I(0), so RHS must be I(0) as well.
\item That must mean that $β'y_{t-1}$ is I(0)
\item $r$ is called the ``cointegrating rank'' of the system
\item If you knew $β$ this would be straightforward to estimate
  (sometimes you do: log oil prices and log jet fuel prices, for
  example)
\item $α$ and $β$ are not unique (but might be with reasonable
  restrictions)
\end{itemize}

More general error-correction form
\[Δ y_t = a_0 + αβ' y_{t-1} + π(L) Δ y_{t-1} + e_t\]

\paragraph{Interpreting cointegration}
\begin{itemize}
\item We'll sometimes know $β$ (at least under the null), in which
  case the analysis is ``easy''
\item cointegrating vectors can represent long-run equilibria
\end{itemize}

\subsection{Working with known cointegrating vectors.}

Note that ``known'' can mean, as always, ``known under the null
hypothesis of interest.  So this material is at least potentially
relevant even in real applications.

If $β$ is known, estimating the VECM is completely straightforward.
If it is of interest to test whether $y_t$ is cointegrated for a
particular value of $β$, we can define $u_t = β'y_t$.
\begin{itemize}
\item If $y_t$ is cointegrated (for that value of $β$), then $β'y_t$
  is $I(0)$.
\item If it is not cointegrated (for that value of $β$), then $β'y_t$
  has a unit root.
\end{itemize}
We can then test for a unit root as usual.

\subsection{Working with unknown cointegrating vectors.}

This is the Engle-Granger approach; let's say for now that $r$ is 1
and that $n$ (the number of series) is 3, and furthermore, suppose
that we know the first variable has to be in the cointegrating
relationship, so $β₁ ≠ 0$.  If we define $v_t = β'y_t$, we can divide
by $β₁$ to get the relationship
\begin{equation}\label{eq:1}
  y_{1t} = \tfrac{1}{β₁} v_t - \tfrac{β₂}{β₁} y_{2t} - \tfrac{β₃}{β₁} y_{3t}.
\end{equation}
We can assume without loss of generality that $β₁=1$ (having already
assumed that it's nonzero).  So a reasonable question is, can we
estimate the cointegrating relationship by estimating~\eqref{eq:1}
with OLS?  And if we do, is the estimation error in this step going to
affect subsequent inference?

We'll do a simple version of the math.  Just as in the stochastic
integration section, we're going to have to scale at a rate different
than $\sqrt{T}$ for a nondegenerate limit distribution.  We have
(dropping the first element of $β$ since it's assumed to be 1)
\begin{equation*}
  T(\hat β - β)
  = \Big( \tfrac{1}{T²} ∑_{t=1}^T e₂'y_t y_t'e₂ \Big)^{-1}
    \tfrac{1}{T} ∑_{t=1}^T e₂'y_t Δy_t'e₁
\end{equation*}
where
\begin{align*}
  e₁ &= \begin{pmatrix} 1 \\ 0 \\ 0 \end{pmatrix} &
  e₂ &= \begin{pmatrix} 0 & 0 \\ 1 & 0 \\ 0 & 1 \end{pmatrix}
\end{align*}
are used to pick off the regressand and the two regressors as in the
previous sections.

Now, since
\[\tfrac{1}{T} y_{[λT]} y_{[λT]}' →^d Σ^{1/2} W(λ)W(λ)' Σ^{1/2\prime},\]
and
\begin{align*}
  \tfrac{1}{T} ∑_{t=1}^T y_t Δy_t
  &= \tfrac{1}{T} ∑_{t=1}^T y_{t-1} Δy_t' + \tfrac{1}{T} ∑_{t=1}^T Δy_t Δy_t' \\
  &→^d Σ^{1/2} ∫₀¹ W(λ) dW(λ) Σ^{1/2\prime} + Σ
\end{align*}
(assuming MDS) this converges to something that's $O_p(1)$, so it's
super-consistent.

An implication of super-consistency is that we can ignore estimation
error in $β$ when we do inference.
\begin{align*}
  \sqrt{T} (\hat α \hat β' - Π)
  &= \sqrt{T} ((\hat α - α + α) (\hat β - β + β)' - αβ') \\
  &= \sqrt{T} ((\hat α - α) β' + (\hat α - α) (\hat β - β) + α (\hat β - β)') \\
  &= \sqrt{T} (\hat α - α) β' + o_p(1).
\end{align*}
Without superconsistency, $\sqrt{T} α (\hat β - β)'$ would not be $o_p(1)$.

So the idea here would be to estimate $β$ first by OLS, then plug in
the estimate as though $β$ were known and proceed as before.

The obvious problem is that this all falls apart if it happens that
$β₁=0$.  Since nothing about the above argument really requires that
it be the \textit{first} element that's known to belong in the
relationship, the issue is how to proceed if we aren't necessarily
sure that any of the elements of $y_t$ is guaranteed to be part of the
cointegrating relationship.

In that case, you'd use Johansen's reduced rank regression; fit the
model with OLS (or MLE) under the constraint that $Π = αβ'$ has
reduced rank $r$.  Operationally, this requires
\begin{itemize}
\item First, test the null that $Π = 0$ in 
  \[ Δy_t = a₀ Π y_{t-1} + π(L) Δy_{t-1} + e_t, \] i.e. test the null
  hypothesis that there is no cointegration (we've already
  decided/determined that $y_t$ is $I(1)$.
\item If you reject the first null, test the null that $r=1$ against
  the alternative that $r>1$.
\item If you reject, continue testing $r=j$ vs $r>j$ for
  $j=2,3,…,n-1$.
\item When you finally fail to reject, set that value for $r$.
\end{itemize}

Obviously, these pre-testing procedures are really problematic in
finite samples.

\section{First-differenced cointegrated processes do not have VAR
  representations.}

Suppose we wanted to just estimate the first differenced model
(i.e. ignore the cointegration)

$Δy_t$ is stationary, so can't we just invoke Wold representation
theorem:
\[Δ y_t = C(L) e_t\]
Use Beveridge-Nelson decomposition ($C^*_j = - \sum_{s=j+1}^∞ C_s$)
\[Δy_t = C(1) e_t + C^*(L) (e_t - e_{t-1})\]
so
\[y_t = y_0 + \sum_{t=0}^t Δ y_t = y_0 + C(1) w_t + C^*(L) e_t\]
w/ $w_t = \sum_{s=0}^t e_t$ (a unit root process)

Now, cointegration implies that $β'y_t$ is I(0), so
\[β'y_0 + β'C(1) w_t + β'C^*(L) e_t\]

must be I(0) as well, which only happens if the $w_t$ term is a.s. zero,
so we need
\[β'C(1) = 0\] as a consequence of cointegration (and vice
versa). This is actually a big deal. remember that for an MA(∞) to be
invertible, we need the solutions to $\det(C(z)) = 0$ to all be
outside the unit circle, which we just ruled out.  So $Δ y_t$
\textbf{does not have a VAR representation}

Also, $t^{-1/2} y_t$ has limiting variance of $C(1) Σ C(1)'$, which
means that $\avar(T^{-1/2} ∑_{t=1}^T Δy_t)$ has the same asymptotic
variance, which doesn't have full rank.

%%% Local Variables: 
%%% mode: latex
%%% TeX-master: "../textbook"
%%% End: 
\appendix
% Copyright © 2013, authors of the "Notes on Macroeconometrics"
% textbook; a complete list of authors is available in the file
% AUTHORS.tex.

% Permission is granted to copy, distribute and/or modify this
% document under the terms of the GNU Free Documentation License,
% Version 1.3 or any later version published by the Free Software
% Foundation; with no Invariant Sections, no Front-Cover Texts, and no
% Back-Cover Texts.  A copy of the license is included in the file
% LICENSE.tex and is also available online at
% <http://www.gnu.org/copyleft/fdl.html>.

\part*{List of Authors}%
\addcontentsline{toc}{part}{Appendix A: List of Authors}

The following is a list of the contributors to the Econometrics Free
Library Project's \textit{Notes on Macroeconometrics}, in order of
their date of first involvement (yes, I'm aware that it's a little
ridiculous to have this as a separate file when there is only a single
contributor, but let's dream big, shall we).

\begin{description}
\item[2013-04-27] Gray Calhoun, \email{gcalhoun@iastate.edu}
\end{description}

%%% Local Variables: 
%%% mode: latex
%%% TeX-master: "textbook"
%%% End: 

% GNU Free Documentation License
\setcounter{section}{-1}%
\renewcommand\thesection{\arabic{section}}%

\noindent Version 1.3, 3 November 2008

\noindent Copyright \textcopyright\ 2000, 2001, 2002, 2007, 2008 Free
Software Foundation, Inc.

\noindent \url{http://fsf.org/}
 
\noindent Everyone is permitted to copy and distribute verbatim copies
of this license document, but changing it is not allowed.

\section{Preamble}

The purpose of this License is to make a manual, textbook, or other
functional and useful document ``free'' in the sense of freedom: to
assure everyone the effective freedom to copy and redistribute it,
with or without modifying it, either commercially or noncommercially.
Secondarily, this License preserves for the author and publisher a way
to get credit for their work, while not being considered responsible
for modifications made by others.

This License is a kind of ``copyleft'', which means that derivative
works of the document must themselves be free in the same sense.  It
complements the GNU General Public License, which is a copyleft
license designed for free software.

We have designed this License in order to use it for manuals for free
software, because free software needs free documentation: a free
program should come with manuals providing the same freedoms that the
software does.  But this License is not limited to software manuals;
it can be used for any textual work, regardless of subject matter or
whether it is published as a printed book.  We recommend this License
principally for works whose purpose is instruction or reference.


\section{APPLICABILITY AND DEFINITIONS}

This License applies to any manual or other work, in any medium, that
contains a notice placed by the copyright holder saying it can be
distributed under the terms of this License.  Such a notice grants a
world-wide, royalty-free license, unlimited in duration, to use that
work under the conditions stated herein.  The ``\textbf{Document}'',
below, refers to any such manual or work.  Any member of the public is
a licensee, and is addressed as ``\textbf{you}''.  You accept the
license if you copy, modify or distribute the work in a way requiring
permission under copyright law.

A ``\textbf{Modified Version}'' of the Document means any work containing the
Document or a portion of it, either copied verbatim, or with
modifications and/or translated into another language.

A ``\textbf{Secondary Section}'' is a named appendix or a front-matter section of
the Document that deals exclusively with the relationship of the
publishers or authors of the Document to the Document's overall subject
(or to related matters) and contains nothing that could fall directly
within that overall subject.  (Thus, if the Document is in part a
textbook of mathematics, a Secondary Section may not explain any
mathematics.)  The relationship could be a matter of historical
connection with the subject or with related matters, or of legal,
commercial, philosophical, ethical or political position regarding
them.

The ``\textbf{Invariant Sections}'' are certain Secondary Sections whose titles
are designated, as being those of Invariant Sections, in the notice
that says that the Document is released under this License.  If a
section does not fit the above definition of Secondary then it is not
allowed to be designated as Invariant.  The Document may contain zero
Invariant Sections.  If the Document does not identify any Invariant
Sections then there are none.

The ``\textbf{Cover Texts}'' are certain short passages of text that are listed,
as Front-Cover Texts or Back-Cover Texts, in the notice that says that
the Document is released under this License.  A Front-Cover Text may
be at most 5 words, and a Back-Cover Text may be at most 25 words.

A ``\textbf{Transparent}'' copy of the Document means a machine-readable copy,
represented in a format whose specification is available to the
general public, that is suitable for revising the document
straightforwardly with generic text editors or (for images composed of
pixels) generic paint programs or (for drawings) some widely available
drawing editor, and that is suitable for input to text formatters or
for automatic translation to a variety of formats suitable for input
to text formatters.  A copy made in an otherwise Transparent file
format whose markup, or absence of markup, has been arranged to thwart
or discourage subsequent modification by readers is not Transparent.
An image format is not Transparent if used for any substantial amount
of text.  A copy that is not ``Transparent'' is called ``\textbf{Opaque}''.

Examples of suitable formats for Transparent copies include plain
ASCII without markup, Texinfo input format, LaTeX input format, SGML
or XML using a publicly available DTD, and standard-conforming simple
HTML, PostScript or PDF designed for human modification.  Examples of
transparent image formats include PNG, XCF and JPG.  Opaque formats
include proprietary formats that can be read and edited only by
proprietary word processors, SGML or XML for which the DTD and/or
processing tools are not generally available, and the
machine-generated HTML, PostScript or PDF produced by some word
processors for output purposes only.

The ``\textbf{Title Page}'' means, for a printed book, the title page itself,
plus such following pages as are needed to hold, legibly, the material
this License requires to appear in the title page.  For works in
formats which do not have any title page as such, ``Title Page'' means
the text near the most prominent appearance of the work's title,
preceding the beginning of the body of the text.

The ``\textbf{publisher}'' means any person or entity that distributes
copies of the Document to the public.

A section ``\textbf{Entitled XYZ}'' means a named subunit of the Document whose
title either is precisely XYZ or contains XYZ in parentheses following
text that translates XYZ in another language.  (Here XYZ stands for a
specific section name mentioned below, such as ``\textbf{Acknowledgements}'',
``\textbf{Dedications}'', ``\textbf{Endorsements}'', or ``\textbf{History}''.)  
To ``\textbf{Preserve the Title}''
of such a section when you modify the Document means that it remains a
section ``Entitled XYZ'' according to this definition.

The Document may include Warranty Disclaimers next to the notice which
states that this License applies to the Document.  These Warranty
Disclaimers are considered to be included by reference in this
License, but only as regards disclaiming warranties: any other
implication that these Warranty Disclaimers may have is void and has
no effect on the meaning of this License.


\section{VERBATIM COPYING}

You may copy and distribute the Document in any medium, either
commercially or noncommercially, provided that this License, the
copyright notices, and the license notice saying this License applies
to the Document are reproduced in all copies, and that you add no other
conditions whatsoever to those of this License.  You may not use
technical measures to obstruct or control the reading or further
copying of the copies you make or distribute.  However, you may accept
compensation in exchange for copies.  If you distribute a large enough
number of copies you must also follow the conditions in section~3.

You may also lend copies, under the same conditions stated above, and
you may publicly display copies.


\section{COPYING IN QUANTITY}

If you publish printed copies (or copies in media that commonly have
printed covers) of the Document, numbering more than 100, and the
Document's license notice requires Cover Texts, you must enclose the
copies in covers that carry, clearly and legibly, all these Cover
Texts: Front-Cover Texts on the front cover, and Back-Cover Texts on
the back cover.  Both covers must also clearly and legibly identify
you as the publisher of these copies.  The front cover must present
the full title with all words of the title equally prominent and
visible.  You may add other material on the covers in addition.
Copying with changes limited to the covers, as long as they preserve
the title of the Document and satisfy these conditions, can be treated
as verbatim copying in other respects.

If the required texts for either cover are too voluminous to fit
legibly, you should put the first ones listed (as many as fit
reasonably) on the actual cover, and continue the rest onto adjacent
pages.

If you publish or distribute Opaque copies of the Document numbering
more than 100, you must either include a machine-readable Transparent
copy along with each Opaque copy, or state in or with each Opaque copy
a computer-network location from which the general network-using
public has access to download using public-standard network protocols
a complete Transparent copy of the Document, free of added material.
If you use the latter option, you must take reasonably prudent steps,
when you begin distribution of Opaque copies in quantity, to ensure
that this Transparent copy will remain thus accessible at the stated
location until at least one year after the last time you distribute an
Opaque copy (directly or through your agents or retailers) of that
edition to the public.

It is requested, but not required, that you contact the authors of the
Document well before redistributing any large number of copies, to give
them a chance to provide you with an updated version of the Document.


\section{MODIFICATIONS}

You may copy and distribute a Modified Version of the Document under
the conditions of sections 2 and 3 above, provided that you release
the Modified Version under precisely this License, with the Modified
Version filling the role of the Document, thus licensing distribution
and modification of the Modified Version to whoever possesses a copy
of it.  In addition, you must do these things in the Modified Version:

\begin{itemize}
\item[A.] 
   Use in the Title Page (and on the covers, if any) a title distinct
   from that of the Document, and from those of previous versions
   (which should, if there were any, be listed in the History section
   of the Document).  You may use the same title as a previous version
   if the original publisher of that version gives permission.
   
\item[B.]
   List on the Title Page, as authors, one or more persons or entities
   responsible for authorship of the modifications in the Modified
   Version, together with at least five of the principal authors of the
   Document (all of its principal authors, if it has fewer than five),
   unless they release you from this requirement.
   
\item[C.]
   State on the Title page the name of the publisher of the
   Modified Version, as the publisher.
   
\item[D.]
   Preserve all the copyright notices of the Document.
   
\item[E.]
   Add an appropriate copyright notice for your modifications
   adjacent to the other copyright notices.
   
\item[F.]
   Include, immediately after the copyright notices, a license notice
   giving the public permission to use the Modified Version under the
   terms of this License, in the form shown in the Addendum below.
   
\item[G.]
   Preserve in that license notice the full lists of Invariant Sections
   and required Cover Texts given in the Document's license notice.
   
\item[H.]
   Include an unaltered copy of this License.
   
\item[I.]
   Preserve the section Entitled ``History'', Preserve its Title, and add
   to it an item stating at least the title, year, new authors, and
   publisher of the Modified Version as given on the Title Page.  If
   there is no section Entitled ``History'' in the Document, create one
   stating the title, year, authors, and publisher of the Document as
   given on its Title Page, then add an item describing the Modified
   Version as stated in the previous sentence.
   
\item[J.]
   Preserve the network location, if any, given in the Document for
   public access to a Transparent copy of the Document, and likewise
   the network locations given in the Document for previous versions
   it was based on.  These may be placed in the ``History'' section.
   You may omit a network location for a work that was published at
   least four years before the Document itself, or if the original
   publisher of the version it refers to gives permission.
   
\item[K.]
   For any section Entitled ``Acknowledgements'' or ``Dedications'',
   Preserve the Title of the section, and preserve in the section all
   the substance and tone of each of the contributor acknowledgements
   and/or dedications given therein.
   
\item[L.]
   Preserve all the Invariant Sections of the Document,
   unaltered in their text and in their titles.  Section numbers
   or the equivalent are not considered part of the section titles.
   
\item[M.]
   Delete any section Entitled ``Endorsements''.  Such a section
   may not be included in the Modified Version.
   
\item[N.]
   Do not retitle any existing section to be Entitled ``Endorsements''
   or to conflict in title with any Invariant Section.
   
\item[O.]
   Preserve any Warranty Disclaimers.
\end{itemize}

If the Modified Version includes new front-matter sections or
appendices that qualify as Secondary Sections and contain no material
copied from the Document, you may at your option designate some or all
of these sections as invariant.  To do this, add their titles to the
list of Invariant Sections in the Modified Version's license notice.
These titles must be distinct from any other section titles.

You may add a section Entitled ``Endorsements'', provided it contains
nothing but endorsements of your Modified Version by various
parties---for example, statements of peer review or that the text has
been approved by an organization as the authoritative definition of a
standard.

You may add a passage of up to five words as a Front-Cover Text, and a
passage of up to 25 words as a Back-Cover Text, to the end of the list
of Cover Texts in the Modified Version.  Only one passage of
Front-Cover Text and one of Back-Cover Text may be added by (or
through arrangements made by) any one entity.  If the Document already
includes a cover text for the same cover, previously added by you or
by arrangement made by the same entity you are acting on behalf of,
you may not add another; but you may replace the old one, on explicit
permission from the previous publisher that added the old one.

The author(s) and publisher(s) of the Document do not by this License
give permission to use their names for publicity for or to assert or
imply endorsement of any Modified Version.


\section{COMBINING DOCUMENTS}

You may combine the Document with other documents released under this
License, under the terms defined in section~4 above for modified
versions, provided that you include in the combination all of the
Invariant Sections of all of the original documents, unmodified, and
list them all as Invariant Sections of your combined work in its
license notice, and that you preserve all their Warranty Disclaimers.

The combined work need only contain one copy of this License, and
multiple identical Invariant Sections may be replaced with a single
copy.  If there are multiple Invariant Sections with the same name but
different contents, make the title of each such section unique by
adding at the end of it, in parentheses, the name of the original
author or publisher of that section if known, or else a unique number.
Make the same adjustment to the section titles in the list of
Invariant Sections in the license notice of the combined work.

In the combination, you must combine any sections Entitled ``History''
in the various original documents, forming one section Entitled
``History''; likewise combine any sections Entitled ``Acknowledgements'',
and any sections Entitled ``Dedications''.  You must delete all sections
Entitled ``Endorsements''.

\section{COLLECTIONS OF DOCUMENTS}

You may make a collection consisting of the Document and other documents
released under this License, and replace the individual copies of this
License in the various documents with a single copy that is included in
the collection, provided that you follow the rules of this License for
verbatim copying of each of the documents in all other respects.

You may extract a single document from such a collection, and distribute
it individually under this License, provided you insert a copy of this
License into the extracted document, and follow this License in all
other respects regarding verbatim copying of that document.


\section{AGGREGATION WITH INDEPENDENT WORKS}

A compilation of the Document or its derivatives with other separate
and independent documents or works, in or on a volume of a storage or
distribution medium, is called an ``aggregate'' if the copyright
resulting from the compilation is not used to limit the legal rights
of the compilation's users beyond what the individual works permit.
When the Document is included in an aggregate, this License does not
apply to the other works in the aggregate which are not themselves
derivative works of the Document.

If the Cover Text requirement of section~3 is applicable to these
copies of the Document, then if the Document is less than one half of
the entire aggregate, the Document's Cover Texts may be placed on
covers that bracket the Document within the aggregate, or the
electronic equivalent of covers if the Document is in electronic form.
Otherwise they must appear on printed covers that bracket the whole
aggregate.


\section{TRANSLATION}

Translation is considered a kind of modification, so you may
distribute translations of the Document under the terms of section~4.
Replacing Invariant Sections with translations requires special
permission from their copyright holders, but you may include
translations of some or all Invariant Sections in addition to the
original versions of these Invariant Sections.  You may include a
translation of this License, and all the license notices in the
Document, and any Warranty Disclaimers, provided that you also include
the original English version of this License and the original versions
of those notices and disclaimers.  In case of a disagreement between
the translation and the original version of this License or a notice
or disclaimer, the original version will prevail.

If a section in the Document is Entitled ``Acknowledgements'',
``Dedications'', or ``History'', the requirement (section~4) to Preserve
its Title (section~1) will typically require changing the actual
title.


\section{TERMINATION}

You may not copy, modify, sublicense, or distribute the Document
except as expressly provided under this License.  Any attempt
otherwise to copy, modify, sublicense, or distribute it is void, and
will automatically terminate your rights under this License.

However, if you cease all violation of this License, then your license
from a particular copyright holder is reinstated (a) provisionally,
unless and until the copyright holder explicitly and finally
terminates your license, and (b) permanently, if the copyright holder
fails to notify you of the violation by some reasonable means prior to
60 days after the cessation.

Moreover, your license from a particular copyright holder is
reinstated permanently if the copyright holder notifies you of the
violation by some reasonable means, this is the first time you have
received notice of violation of this License (for any work) from that
copyright holder, and you cure the violation prior to 30 days after
your receipt of the notice.

Termination of your rights under this section does not terminate the
licenses of parties who have received copies or rights from you under
this License.  If your rights have been terminated and not permanently
reinstated, receipt of a copy of some or all of the same material does
not give you any rights to use it.


\section{REVISIONS OF THIS LICENSE}

The Free Software Foundation may publish new, revised versions
of the GNU Free Documentation License from time to time.  Such new
versions will be similar in spirit to the present version, but may
differ in detail to address new problems or concerns.  See
\url{http://www.gnu.org/copyleft/}.

Each version of the License is given a distinguishing version number.
If the Document specifies that a particular numbered version of this
License ``or any later version'' applies to it, you have the option of
following the terms and conditions either of that specified version or
of any later version that has been published (not as a draft) by the
Free Software Foundation.  If the Document does not specify a version
number of this License, you may choose any version ever published (not
as a draft) by the Free Software Foundation.  If the Document
specifies that a proxy can decide which future versions of this
License can be used, that proxy's public statement of acceptance of a
version permanently authorizes you to choose that version for the
Document.

\section{RELICENSING}

``Massive Multiauthor Collaboration Site'' (or ``MMC Site'') means any
World Wide Web server that publishes copyrightable works and also
provides prominent facilities for anybody to edit those works.  A
public wiki that anybody can edit is an example of such a server.  A
``Massive Multiauthor Collaboration'' (or ``MMC'') contained in the
site means any set of copyrightable works thus published on the MMC
site.

``CC-BY-SA'' means the Creative Commons Attribution-Share Alike 3.0
license published by Creative Commons Corporation, a not-for-profit
corporation with a principal place of business in San Francisco,
California, as well as future copyleft versions of that license
published by that same organization.

``Incorporate'' means to publish or republish a Document, in whole or
in part, as part of another Document.

An MMC is ``eligible for relicensing'' if it is licensed under this
License, and if all works that were first published under this License
somewhere other than this MMC, and subsequently incorporated in whole
or in part into the MMC, (1) had no cover texts or invariant sections,
and (2) were thus incorporated prior to November 1, 2008.

The operator of an MMC Site may republish an MMC contained in the site
under CC-BY-SA on the same site at any time before August 1, 2009,
provided the MMC is eligible for relicensing.

\section*{ADDENDUM: How to use this License for your documents}
\addcontentsline{toc}{section}{ADDENDUM: How to use this License for your documents}

To use this License in a document you have written, include a copy of
the License in the document and put the following copyright and
license notices just after the title page:

\bigskip
\begin{quote}
    Copyright \textcopyright\ YEAR  YOUR NAME.
    Permission is granted to copy, distribute and/or modify this document
    under the terms of the GNU Free Documentation License, Version 1.3
    or any later version published by the Free Software Foundation;
    with no Invariant Sections, no Front-Cover Texts, and no Back-Cover Texts.
    A copy of the license is included in the section entitled ``GNU
    Free Documentation License''.
\end{quote}
\bigskip
    
If you have Invariant Sections, Front-Cover Texts and Back-Cover Texts,
replace the ``with \dots\ Texts.''\ line with this:

\bigskip
\begin{quote}
    with the Invariant Sections being LIST THEIR TITLES, with the
    Front-Cover Texts being LIST, and with the Back-Cover Texts being LIST.
\end{quote}
\bigskip
    
If you have Invariant Sections without Cover Texts, or some other
combination of the three, merge those two alternatives to suit the
situation.

If your document contains nontrivial examples of program code, we
recommend releasing these examples in parallel under your choice of
free software license, such as the GNU General Public License,
to permit their use in free software.

%%% Local Variables:
%%% mode: latex
%%% TeX-master: "../macroeconometrics"
%%% End:


\part*{References}
\addcontentsline{toc}{part}{References}
\bibliography{tex/references}
\end{document}

%%% Local Variables:
%%% mode: latex
%%% TeX-master: "textbook"
%%% End: