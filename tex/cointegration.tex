% Copyright © 2013, Gray Calhoun.  Permission is granted to copy,
% distribute and/or modify this document under the terms of the GNU
% Free Documentation License, Version 1.3 or any later version
% published by the Free Software Foundation; with no Invariant
% Sections, no Front-Cover Texts, and no Back-Cover Texts.  A copy of
% the license is included in the section entitled "GNU Free
% Documentation License."

\chapter{Cointegration}

\begin{itemize}
\item Reading: Davidson (2012); Chapter 10 of Hayashi; Hamilton
\end{itemize}

Look at VAR(1): $y_t = a_0 + A y_{t-1} + e_t$; remember that if we
want something like stationarity, we need to look at the roots of the
polynomial $det(I - A z) = 0$; nonstationary if there are roots on the
unit circle (ie $|z| = 1$);

possibilities:
\begin{itemize}
\item $n$ roots ($n$ is the number of equations) on the unit circle
\item less than $n$, but positive
\item no roots on unit circle
\end{itemize}

Let $Π = A - I$ and rewrite the VAR as 
\[Δ y_t = a_0 + Π y_{t-1} + e_t\]

$Π$ may not have full rank, but if it has rank $r$, we can always write $Π =
αβ'$ where $α$ and $β$ both are $k × r$ with full rank.
Then \[Δ y_t = a_0 + αβ' y_{t-1} + e_t\]
(essentially, this is the ``Granger representation theorem'')

why do this: 
\begin{itemize}
\item LHS is I(0), so RHS must be I(0) as well.
\item That must mean that $β'y_{t-1}$ is I(0)
\item $r$ is called the ``cointegrating rank'' of the system
\item If you knew $β$ this would be straightforward to estimate
  (sometimes you do: log oil prices and log jet fuel prices, for
  example)
\item $α$ and $β$ are not unique (but might be with reasonable
  restrictions)
\end{itemize}

More general error-correction form
\[Δ y_t = a_0 + αβ' y_{t-1} + π(L) Δ y_{t-1} + e_t\]

\section{Interesting features of VECM}

Suppose we wanted to just estimate the first differenced model
(i.e.~ignore the cointegration)

$Δ y_t$ is stationary, so can't we just invoke Wold representation
theorem:
\[Δ y_t = C(L) e_t\]
Use Beveridge-Nelson decomposition ($C^*_j = - \sum_{s=j+1}^∞ C_s$)
\[Δ y_t = C(1) e_t + C^*(L) (e_t - e_{t-1})\]
so
\[y_t = y_0 + \sum_{t=0}^t Δ y_t = y_0 + C(1) w_t + C^*(L) e_t\]
w/ $w_t = \sum_{s=0}^t e_t$ (a unit root process)

Now, cointegration implies that $β'y_t$ is I(0), so
\[β'y_0 + β'C(1) w_t + β'C^*(L) e_t\]

must be I(0) as well, which only happens if the $w_t$ term is a.s. zero,
so we need
\[β'C(1) = 0\]
as a consequence of cointegration (and vice versa). This is actually a
big deal. remember that for an MA(∞) to be invertible, we need $det(C(z))
≠ 0$ for all (possibly complex) $z$ not on the unit circle. So $Δ y_t$
\textbf{does not have a VAR representation}

Also, $t^{-1/2} y_t$ has asymptotic variance of $C(1) Σ C(1)'$

\section{Interpreting}

\begin{itemize}
\item We'll sometimes know $β$ (at least under the null), in which
  case the analysis is ``easy''
\item cointegrating vectors can represent long-run equilibria
\end{itemize}

%%% Local Variables: 
%%% mode: latex
%%% TeX-master: "../textbook"
%%% End: 