% Copyright © 2013-2014 Gray Calhoun

% Permission is granted to copy, distribute and/or modify this
% document under the terms of the GNU Free Documentation License,
% Version 1.3 or any later version published by the Free Software
% Foundation; with no Invariant Sections, no Front-Cover Texts, and no
% Back-Cover Texts.  A copy of the license is included in the file
% LICENSE.tex and is also available online at
% <http://www.gnu.org/copyleft/fdl.html>.

\section{Cointegration I}

\subsection{Introduction}

Look at VAR(1): $y_t = a_0 + A y_{t-1} + e_t$; remember that if we
want something like stationarity, we need to look at the roots of the
polynomial $\det(I - A z) = 0$; nonstationary if there are roots on
the unit circle (ie $|z| = 1$);

possibilities:
\begin{itemize}
\item $n$ roots ($n$ is the number of equations) on the unit circle
\item less than $n$, but positive
\item no roots on unit circle
\end{itemize}

We haven't formally defined what it means for a process to be $I(0)$
yet, but it captures what we usually mean when we say that a series is
stationary.  If a series is stationary, we typically expect that it
obeys a CLT.  But it is straightforward to write down stationary
series that don't (for example, let $u_t$ be an i.i.d. sequence; then
$\Delta u_t$ is stationary but obviously does not obey a CLT).

So we say that the series $u_t$ is $I(0)$ if it has the MA
representation $u_t = \Phi(L) \epsilon_t$ where $\epsilon_t$ is white noise, $\Phi(1) \neq
0$, and $\sum_{j=0}^\infty | \Phi_j |$ is finite (element-by-element in the case
of multivariate processes).

Now we extend the definition to $I(d)$ recursively: for $d > 0$, $u_t$
is $I(d)$ if $\Delta u_t$ is $I(d-1)$ and is $I(-d)$ if $\sum_{j=-\infty}^\infty u_t$ is
$I(1-d)$.

\paragraph{Cointegration setup}
Let $\Pi = A - I$ and rewrite the VAR as
\[\Delta  y_t = a_0 + \Pi y_{t-1} + e_t\]

$\Pi$ may not have full rank, but if it has rank $r$, we can always write $\Pi =
\alpha\beta'$ where $\alpha$ and $\beta$ both are $k × r$ with full rank.
Then \[\Delta y_t = a_0 + \alpha\beta'y_{t-1} + e_t\]
(essentially, this is the ``Granger representation theorem'')

\paragraph{Why must it be $\beta'y_t$ that's $I(0)$?} The intuition's
easiest to see when $r=1$.  In that case, $\alpha$ becomes just a vector of
constants, so it can't possibly remove the stochastic trend.

why do this:
\begin{itemize}
\item LHS is I(0), so RHS must be I(0) as well.
\item That must mean that $\beta'y_{t-1}$ is I(0)
\item $r$ is called the ``cointegrating rank'' of the system
\item If you knew $\beta$ this would be straightforward to estimate
  (sometimes you do: log oil prices and log jet fuel prices, for
  example)
\item $\alpha$ and $\beta$ are not unique (but might be with reasonable
  restrictions)
\end{itemize}

More general error-correction form
\[\Delta  y_t = a_0 + \alpha\beta' y_{t-1} + \pi(L) \Delta  y_{t-1} + e_t\]

\paragraph{Interpreting cointegration}
\begin{itemize}
\item We'll sometimes know $\beta$ (at least under the null), in which
  case the analysis is ``easy''
\item cointegrating vectors can represent long-run equilibria
\end{itemize}

\subsubsection{Working with known cointegrating vectors.}

Note that ``known'' can mean, as always, ``known under the null
hypothesis of interest.  So this material is at least potentially
relevant even in real applications.

If $\beta$ is known, estimating the VECM is completely straightforward.
If it is of interest to test whether $y_t$ is cointegrated for a
particular value of $\beta$, we can define $u_t = \beta'y_t$.
\begin{itemize}
\item If $y_t$ is cointegrated (for that value of $\beta$), then $\beta'y_t$
  is $I(0)$.
\item If it is not cointegrated (for that value of $\beta$), then $\beta'y_t$
  has a unit root.
\end{itemize}
We can then test for a unit root as usual.

\subsubsection{Working with unknown cointegrating vectors.}

This is the Engle-Granger approach; let's say for now that $r$ is 1
and that $n$ (the number of series) is 3, and furthermore, suppose
that we know the first variable has to be in the cointegrating
relationship, so $\beta_1 \neq 0$.  If we define $v_t = \beta'y_t$, we can divide
by $\beta_1$ to get the relationship
\begin{equation}\label{eq:1}
  y_{1t} = \tfrac{1}{\beta_1} v_t - \tfrac{\beta₂}{\beta_1} y_{2t} - \tfrac{\beta₃}{\beta_1} y_{3t}.
\end{equation}
We can assume without loss of generality that $\beta_1=1$ (having already
assumed that it's nonzero).  So a reasonable question is, can we
estimate the cointegrating relationship by estimating~\eqref{eq:1}
with OLS?  And if we do, is the estimation error in this step going to
affect subsequent inference?

We'll do a simple version of the math.  Just as in the stochastic
integration section, we're going to have to scale at a rate different
than $\sqrt{T}$ for a nondegenerate limit distribution.  We have
(dropping the first element of $\beta$ since it's assumed to be 1)
\begin{equation*}
  T(\hat \beta - \beta)
  = \Big( \tfrac{1}{T²} \sum_{t=1}^T e₂'y_t y_t'e₂ \Big)^{-1}
    \tfrac{1}{T} \sum_{t=1}^T e₂'y_t \Delta y_t'e_1
\end{equation*}
where
\begin{align*}
  e_1 &= \begin{pmatrix} 1 \\ 0 \\ 0 \end{pmatrix} &
  e₂ &= \begin{pmatrix} 0 & 0 \\ 1 & 0 \\ 0 & 1 \end{pmatrix}
\end{align*}
are used to pick off the regressand and the two regressors as in the
previous sections.

Now, since
\[\tfrac{1}{T} y_{[\lambda T]} y_{[\lambda T]}' \to^d \Sigma^{1/2} W(\lambda)W(\lambda)' \Sigma^{1/2\prime},\]
and
\begin{align*}
  \tfrac{1}{T} \sum_{t=1}^T y_t \Delta y_t
  &= \tfrac{1}{T} \sum_{t=1}^T y_{t-1} \Delta y_t' + \tfrac{1}{T} \sum_{t=1}^T \Delta y_t \Delta y_t' \\
  &\to^d \Sigma^{1/2} \int_0^1 W(\lambda) dW(\lambda) \Sigma^{1/2\prime} + \Sigma
\end{align*}
(assuming MDS) this converges to something that's $O_p(1)$, so it's
super-consistent.

An implication of super-consistency is that we can ignore estimation
error in $\beta$ when we do inference.
\begin{align*}
  \sqrt{T} (\hat \alpha \hat \beta' - \Pi)
  &= \sqrt{T} ((\hat \alpha - \alpha + \alpha) (\hat \beta - \beta + \beta)' - \alpha\beta') \\
  &= \sqrt{T} ((\hat \alpha - \alpha) \beta' + (\hat \alpha - \alpha) (\hat \beta - \beta) + \alpha (\hat \beta - \beta)') \\
  &= \sqrt{T} (\hat \alpha - \alpha) \beta' + o_p(1).
\end{align*}
Without superconsistency, $\sqrt{T} \alpha (\hat \beta - \beta)'$ would not be $o_p(1)$.

So the idea here would be to estimate $\beta$ first by OLS, then plug in
the estimate as though $\beta$ were known and proceed as before.

The obvious problem is that this all falls apart if it happens that
$\beta_1=0$.  Since nothing about the above argument really requires that
it be the \textit{first} element that's known to belong in the
relationship, the issue is how to proceed if we aren't necessarily
sure that any of the elements of $y_t$ is guaranteed to be part of the
cointegrating relationship.

In that case, you'd use Johansen's reduced rank regression; fit the
model with OLS (or MLE) under the constraint that $\Pi = \alpha\beta'$ has
reduced rank $r$.  Operationally, this requires
\begin{itemize}
\item First, test the null that $\Pi = 0$ in
  \[ \Delta y_t = a_0 \Pi y_{t-1} + \pi(L) \Delta y_{t-1} + e_t, \] i.e. test the null
  hypothesis that there is no cointegration (we've already
  decided/determined that $y_t$ is $I(1)$.
\item If you reject the first null, test the null that $r=1$ against
  the alternative that $r>1$.
\item If you reject, continue testing $r=j$ vs $r>j$ for
  $j=2,3,\dots,n-1$.
\item When you finally fail to reject, set that value for $r$.
\end{itemize}

Obviously, these pre-testing procedures are really problematic in
finite samples.

\subsection{First-differenced cointegrated processes do not have VAR
  representations.}

Suppose we wanted to just estimate the first differenced model
(i.e. ignore the cointegration)

$\Delta y_t$ is stationary, so can't we just invoke Wold representation
theorem:
\[\Delta  y_t = C(L) e_t\]
Use Beveridge-Nelson decomposition ($C^*_j = - \sum_{s=j+1}^\infty C_s$)
\[\Delta y_t = C(1) e_t + C^*(L) (e_t - e_{t-1})\]
so
\[y_t = y_0 + \sum_{t=0}^t \Delta  y_t = y_0 + C(1) w_t + C^*(L) e_t\]
w/ $w_t = \sum_{s=0}^t e_t$ (a unit root process)

Now, cointegration implies that $\beta'y_t$ is I(0), so
\[\beta'y_0 + \beta'C(1) w_t + \beta'C^*(L) e_t\]

must be I(0) as well, which only happens if the $w_t$ term is a.s. zero,
so we need
\[\beta'C(1) = 0\] as a consequence of cointegration (and vice
versa). This is actually a big deal. remember that for an MA($\infty$) to be
invertible, we need the solutions to $\det(C(z)) = 0$ to all be
outside the unit circle, which we just ruled out.  So $\Delta y_t$
\textbf{does not have a VAR representation}

Also, $t^{-1/2} y_t$ has limiting variance of $C(1) \Sigma C(1)'$, which
means that $\avar(T^{-1/2} \sum_{t=1}^T \Delta y_t)$ has the same asymptotic
variance, which doesn't have full rank.

\subsection{Papers to add from syllabus}
\begin{itemize}
\item \citet{EG87}
\item \citet{Jo88}
\item \citet{Jo91}
\item \citet{Wa94}
\end{itemize}

%%% Local Variables:
%%% mode: latex
%%% TeX-master: "../macroeconometrics"
%%% End:
