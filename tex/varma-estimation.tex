% Copyright © 2013, Gray Calhoun.  Permission is granted to copy,
% distribute and/or modify this document under the terms of the GNU
% Free Documentation License, Version 1.3 or any later version
% published by the Free Software Foundation; with no Invariant
% Sections, no Front-Cover Texts, and no Back-Cover Texts.  A copy of
% the license is included in the section entitled "GNU Free
% Documentation License."

\chapter{Lectures on estimation (VARMA, VAR)}

Textbooks:
\begin{itemize}
\item Chapters 5 of Hamilton
\item Brockwell and Davis
\end{itemize}

\section{Parameter estimation}

\begin{itemize}
\item Method of moments
\item Maximum Likelihood (and conditional MLE)
\end{itemize}

\subsection{Estimation of VAR parameters}

Have the model (again):
\[y_t = μ + ∑_{i=1}^p φ_i y_{t-i} + e_t\] $k$ equations, $k p + 1$
regression coefficients. Under assumptions of causality, we know that
$e_t$ is uncorrelated with $y_s$ for $s < t$, which implies that we
might be able to justify OLS:
\begin{itemize}
\item Let $z_t' = (1 y_{t-1}' ⋯ y_{t-p}')$
\item $Φ = [ μ φ_1 … φ_p ]'$
\item Then $y_t' = z_t'Φ' + e_t'$ and we can do equation-by-equation
  OLS, which is equivalent to
  \[\hat Φ' = (∑_{t=p+1}^T z_t z_t')^{-1} ∑_{t=p+1}^T z_t y_t'\]
\end{itemize}

This is the method of moments estimator as well as the conditional MLE
estimator if you assume that $e_t ∼ N(0, Σ)$ and
condition on $y_1$,…,$y_p$. To do MLE, observe that
\[y_t ∣ z_t ∼ N(Φ z_t, Σ).\]

The joint likelihood can be taken as the product of the conditional
likelihoods:
\[L(Φ, Σ; y_1,…,y_T) = f_T(y_T ∣ y_{T-1},…, y_1) f_{T-1}(y_{T-1} ∣
y_{T-2},…, y_1) ⋯ f_{p+1}(y_{p+1} ∣ y_p,…,y_1) f_p(y_p,…,y_1)\]

Assuming normality and correct specification,
\[f_t(y_t ∣ y_{t-1},…, y_1) = f_t(y_t ∣ z_t).\]

If you want, you can avoid conditioning on $y_p$,…,$y_1$ since the
joint dist of those observations is also normal and we've worked out
how to find the mean and variance.

\subsection{Estimation for VARMA models (MA is a special case)}

Two approaches (we won't go into much detail):

\begin{enumerate}
\item Assume $e_t$ is normal, then $y_1,…,y_T$ is jointly normal and
  you can maximize the likelihood w/rt $Φ$, $Θ$, and $Σ$
\end{enumerate}

\begin{itemize}
\item Amounts to invoking invertibility to derive $f_T(y_t ∣
  y_{t-1},…,y_1)$ efficiently for each $t$ w/out using strict VAR
  structure, then building up the unconditional likelihood from that.
\end{itemize}

\begin{enumerate}
\item Just like for VAR and VMA processes, we can derive equations
  that define the autocovariances at lags $0,…,p+q$ in terms of $Φ$
  and $Θ$; solving for $Φ$ and $Θ$ and plugging in the sample
  autocovariances give the ``Yule-Walker'' estimates.
\end{enumerate}

\begin{itemize}
\item For VARs, this is OLS
\item When there's an MA component, this can be inefficient.
\end{itemize}

\section{Asymptotic theory for time-series}

Look at OLS estimator for VAR(p). As always, two components:
\[\hat Φ' = Φ' + (1/T ∑_{t=p+1}^T z_t z_t')^{-1} 1/T
∑_{t=p+1}^T z_t e_t'\] so, for consistency need the sum of $z$'s to be
$O_p(1)$; second sum to be $o_p(1)$. For asymptotic distribution
(normality) need second summation to obey a CLT.

\begin{itemize}
\item Obviously, if we want to estimate the first or second moments of
  a time-series process, we might want to use the sample moments.
\item Of interest on their own
\item Necessary for consistency and asymptotic normality of MoM
  estimators
\item Just look at univariate for simplicity.
\end{itemize}

\begin{description}
\item[Filtration:]
  A sequence of sigma-fields $\{G_t\}$ is a \emph{filtration} if $G_t
  ⊂ G_{t+1}$ for all $t$
\item[Martingale difference sequence:]
  The sequence of rvs $\{e_t\}$ is an mds w/rt the filtration $\{G_t\}$ if
  $E(e_t ∣ G_{t-1}) = 0$.

  Typically $G_t = σ(e_t, e_{t-1},…)$

  Comes from the definition of a martingale: $y_t$ is a martingale
  w/rt the filtration if $E(y_t ∣ G_{t-1}) = y_{t-1}$. Then $e_t =
  Δy_t$ is an mds.
\end{description}

\begin{itemize}
\item Now, suppose that $e_t$ has finite variance and is an mds w/rt
  $G_t = σ(y_t, y_{t-1},…)$, then we know that the second sum has mean
  zero. Also
  \begin{align*}
    E (1/T ∑ z_t e_t)(1/T ∑ e_s z_s')
    &= 2/T^2 ∑_{s ≤ t} E(e_s e_t z_t z_s' ∣ G_{t-1}) \\
    &= 1/T^2 ∑_t E E( e_t^2 ∣ G_{t-1}) z_t z_t' + 2/T^2 ∑_{s < t} E E(
    e_t ∣ G_{t-1}) e_s z_t z_s' \\
    &= σ^2 1/T^2 ∑_t E z_t z_t'
  \end{align*}

  This converges to zero since $y_t$ has finite variance and Φ-hat
  converges in MSE to $Φ$.
\item Note that if $e_t$ is MDS, so is $z_t e_t$
\end{itemize}

\subsection{Standard MDS CLT (McLeish 1974; Hall and Heyde 1980)}

Suppose that $Z_t$ is a (univariate) MDS and define
\begin{itemize}
\item $U_n^2 = ∑_1^n Z_t^2$,
\item $s_n^2 = E U_n^2$.
\end{itemize}
If
\begin{itemize}
\item $U_n^2 / s_n^2 → 1$ i.p. and
\item $max_{1 ≤ t ≤ n} |Z_n / s_n| → 0$ i.p.
\end{itemize}
then
\[1/s_n ∑_1^n Z_t →^d N(0, 1)\]
or, equivalently,
\[1/U_n ∑ Z_t →^d N(0, 1)\]

For multivariate, let $Ω_n = ∑ E Z_t Z_t'$ and we have $Ω_n^{-1/2} ∑
Z_t → N(0, I)$ under essentially the same conditions.

Apply this to second summation in OLS coefficients and (assuming the
conditions hold) we can see that
\[n^{-1/2} ∑ z_t e_t = ( Ω_n / n )^{1/2} Ω_n^{-1/2} ∑ z_t e_t
→^d N(0,Ω) \]

where $Ω_n = ∑ E( e_t^2 z_t z_t' )$ and $Ω = \lim Ω_n / n$. Under
homoskedasticity, this can be simplified.

If $1/n ∑ z_t z_t' → V$ i.p., then $\lim Ω_n = σ^2 V$ under
homoskedasticity and we have
\[\sqrt{n} ( \hat φ - φ ) →^d N( 0, σ^2 V^{-1} ).\]

Under heteroskedasticity,
\[\sqrt{n} (\hat φ - φ ) →^d N( 0, V^{-1} Ω V^{-1} )\]

\textbf{For the asymptotic distribution of the VAR(p)} obviously it's
joint normal, so the only tricky thing is accounting for all of the
correlations. Rewrite it as
\[
sqrt(n) (\hat Φ' - Φ')
= (1/n ∑_{t=p+1}^n z_t z_t')^{-1} 1/sqrt(n) ∑_{t=p+1}^n z_t e_t'
\]
so
\[\sqrt{n} ( vec(\hat Φ' ) - vec( Φ' ) ) = vec( … )\]

and use the fact that, for conformable matrices $A$, $B$, and $C$,
\[vec( A B C ) = ( C ⊗ A ) vec( B )\]
where $⊗$ is the kronecker product and
\[
C ⊗ A = 
\begin{pmatrix}
  c_{11}A & ⋯ & c_{1n}A \\
  ⋮
  c_{n1}A & ⋯ & c_{nn}A 
\end{pmatrix}
\]
$A = ( 1/n ∑ z_t z_t' )^{-1}$, $B = 1/sqrt{n} ∑ z_t e_t'$, $C = I$ so
$C ⊗ A = diag( A,…, A )$

if $V = plim 1/n ∑ z_t z_t' Ω = plim 1/n ∑ vec( z_t e_t' ) vec( z_t
e_t' )'$, then $\sqrt{n} vec(\hat Φ' - Φ' ) → N( 0, diag(
V^{-1},…, V^{-1} ) Ω diag(V^{-1},…,V^{-1} )$

Note that the block diagonal structure means that if we're interested
in the joint dist. of the coefficients for just a single equation,
it's not affected by those from other equations (only holds when all
of the equations have the same regressors…).

\subsection{Result (in summary)}

Suppose that $y_t$ is a VAR(p),
\begin{itemize}
\item $1/n ∑ z_t z_t' → V$ i.p.
\item $1/n ∑ z_t z_t' e_t^2 → Ω$ i.p.
\item $e_t$ is an mds with finite $4 + δ$ moments ($δ > 0$)
\end{itemize}
Then $\sqrt{T} ( \hat Φ' - Φ' )$ is asymptotically normal (with the
variance we derived).

\subsection{Laws of large numbers}
We still need to talk about the square terms:
\[1/n ∑ z_t z_t'\]
and
\[1/n ∑ vec( z_t e_t' ) vec( z_t e_t' )'\]
\begin{description}
\item[LLN for martingale difference sequences:]
If $z_t$ is uniformly integrable (i.e. a slightly weaker condition than
assuming it has $1 + δ$ moments, $δ > 0$) and MDS, then
\[1/n ∑ z_t →^{L_1} 0\]
\end{description}

MDS for second moments is not usually an implication of Economic
Theory; is not usually a feature of a correctly specified model of the
mean; and is often false in Economic data. So it's worth mentioning
another result:

\begin{description}
\item[LLN for stationary and ergodic sequences:]

  If $\{z_t\}$ is a stationary and ergodic sequence with finite mean
  $μ$, then
  \[1/n ∑ z_t → μ\] (in $L_1$).

  Ergodic: a stationary sequence $\{x_t\}$ is \emph{ergodic} if, for
  any bounded functions $f$ and $g$, as $n → ∞$,
  \[E f( x_t, …, x_{t + k}) g(x_{t + n}, …, x_{ t + n + l} ) → E f(
  x_t, …, x_{ t + k } ) E g(x_{ t + n }, …, x_{ t + n + l} )\]

  \begin{itemize}
  \item remember, you can prove independence by proving this
    factorization for all bounded $f$ and $g$, so this is an ``asymptotic
    independence'' condition;
  \item Sometimes see conditions on how fast the convergence to zero
    happens
  \item Nothing special about stationarity; you'll see other weak
    dependence conditions based on similar ideas (i.e. mixing
    conditions).
  \end{itemize}
\end{description}

\subsection{CLT with serial correlation}

Have a CLT for MDS. But what if we have an MA process?
\[y_t = ∑_{j=0}^∞ θ_j e_{t-j} = θ(L) e_t\]

\begin{itemize}
\item Just focus on the average: $̱\bar y$
\item Obviously, if $θ_j$ decays fast enough, LLN holds.
\item Look at CLT:
  \begin{align*}
    T^{1/2} \bar y
    &= 1/T^{1/2} ∑_{t=1}^T ∑_{j=0}^∞ θ_j e_{t-j} \\
    &= \sqrt{1/T} ( θ_0 e_T +
    ( θ_0 + θ_1 ) e_{T-1} +
    ( θ_0 + θ_1 + θ_2 )  e_{T-2} + ...
    ( θ_0 + ... + θ_{T-1} ) e_1 + ... ) \\
    &= \sqrt{1/T} ( ( θ(1) - ∑_1^∞ θ_j ) e_T +
    ( θ(1) - ∑_2^∞ θ_j ) e_{T-1} + ...
    ( θ(1) - ∑_T^∞ θ_j ) e_1 + ... ) \\
    &= \sqrt{1/T} θ(1) ∑_{t=1}^T e_t + \sqrt{1/T} θ^*(L) e_t
  \end{align*}
  where $θ_j^* = - ∑_{i=j+1}^∞ θ_i$.  Now $θ(1)$ has the bulk of the
  dependence, so the first term obeys a CLT and the second can
  converge to zero i.p.
\end{itemize}

Result: Suppose that $y_t = μ + θ(L) e_t$ where
\begin{itemize}
\item $∑_{s=0}^∞ s |θ_s| < ∞$
\item $1/\sqrt{T} ∑_t e_t → N(0, σ^2)$ in distribution (i.e. any CLT
  holds)
\end{itemize}
Then $\sqrt{T} (\bar y - μ ) → N( 0, σ^2 θ(1)^2 )$ in distribution

\section{Last bits}

Forecasting from VAR:
\begin{itemize}
\item 1-step ahead $\hat y_{T+1} = \hat φ(L) y_T$
\item $k$-step ahead $\hat y_{T+k} = \hat φ(L) y_{T+k-1} = … = ( \hat
  φ(L) )^k y_T$
\end{itemize}
Also see forecasting by just regressing $y_{t+k}$ on $y_t,…,y_{t-p}$

What to do with these approximations?
\begin{itemize}
\item Test as normal
\item Test for Granger causality/exogeneity
  \begin{itemize}
  \item split up our variables into
    \begin{align*}
      y_{1t} &= μ_1 + A_1 x_{1t} + A_2 x_{2t} + e_t \\
      y_{2t} &= μ_2 + B_1 x_{1t} + B_2 x_{2t} + e_t
    \end{align*}
    where the $x_{it}$ contains lags of $y_{it}$.
  \item $y_2$ does not granger-cause $y_1$ if lags of $y_2$ do not
    help predict $y_1$ after accounting for lags of $y_1$ (also means
    that $y_1$ is exogenous w/rt $y_2$).
  \item Means that $A_2 = 0$.
  \item This is something we can test.
  \end{itemize}
\end{itemize}

\subsection{Lag length selection}

\begin{itemize}
\item $AIC_p = \log | Σ_u(p)| + 2 p m^2 / T$
\item $BIC_p = \log | Σ_u(p)| + \log(T) 2 p m^2 / T$
\item $Σ_u(q)$ is the vcv of $y_t - b_1 y_{t-1} - … - y_{t-p}$
\item Choose to minimize the criterion
\item Last term works as a penalty
\end{itemize}

%%% Local Variables: 
%%% mode: latex
%%% TeX-master: "../textbook"
%%% End: 