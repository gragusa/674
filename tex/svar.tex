% Copyright © 2013, Gray Calhoun.  Permission is granted to copy,
% distribute and/or modify this document under the terms of the GNU
% Free Documentation License, Version 1.3 or any later version
% published by the Free Software Foundation; with no Invariant
% Sections, no Front-Cover Texts, and no Back-Cover Texts.  A copy of
% the license is included in the section entitled "GNU Free
% Documentation License."

\chapter{Structural VARs}

Textbook material:
\begin{itemize}
\item Greene 6th edition, Chapter 19
\item Greene 7th edition, Chapter 20
\end{itemize}
Review articles:
\begin{itemize}
\item Kilian's recent (2011?) handbook chapter
\item Stock and Watson (2001) \emph{JEP}. ``Vector Autoregressions''
\item Watson (1994) \emph{Handbook}. ``Vector autoregressions and
  cointegration'' (but it's pretty old at this point).
\end{itemize}
Important background:
\begin{itemize}
\item Sims (1980) Macroeconomics and Reality
\end{itemize}
Other books that students may want to consult:
\begin{itemize}
\item Jim Hamilton (1994), \emph{Time series analysis}.
\end{itemize}

\section{Quick: forecast from VAR(p)}

\begin{itemize}
\item Iterated forecasts
\item Direct forecasts
\end{itemize}

\subsection{Uncertainty}

\begin{itemize}
\item Want confidence intervals associated with these IRFs (draw)
\item δ-method (example for AR(1)): 
  \[y_t = ρ y_{t-1} + e_t,\]
  then we know that 
  \[  y_{t+k} - y_{t-1} = ρ^k Δe_t;\] 
  so we'd like to make a confidence region for
\end{itemize}
\[\sqrt{T} (\hat ρ^1 - ρ^1,…,\hat ρ^k - ρ^k)\]
for any $j$, $\sqrt{T} (\hat ρ^j - ρ^j) → N(0, (j ρ^{j-1}) 2 σ²)$,
which breaks if $ρ = 0$!

\begin{itemize}
\item For matrices, spots where it breaks are less obvious and depend
  on the derivative of the AR coefficients.
\item Bootstrap also used (but suffers from some of the same problems
  as $δ$-method)
\item grid bootstrap (Mikusheva 2012)
\item mostly get pointwise CIs
\end{itemize}

(Add material on Bayesian CIs too)

\section{Bayesian IRFs}

\begin{itemize}
\item Bayesian estimation is becoming \emph{massively} popular in
  Macroeconometrics
\item Basic idea: ``completely'' (in some sense) model our uncertainty
  about the unknown parameter values: $φ$, $Σ$ (in this case)
\item suppose we have 5 observations $Y ∼ bernoulli(p)$
\item \emph{Frequentist statistics}: something like MLE
  \[f_Y(y; p) ∼ ∏ p_i^{y_i} (1 - p_i)^{1 - y_1}\]
  \begin{itemize}
  \item Point estimation; hypothesis test; confidence intervals
    (artificial decision making)
  \item Doesn't work very well for more nuanced decision making
    (imagine that you have four observations: 1, 0, 1, 1)
  \item mean is 3/4; 90\% CI is about 25\% to 97\% (47\% to 90\%?)
  \item Suppose you were going to use this information to bet… how?
  \end{itemize}
\item \emph{Bayesian statistics}: model uncertainty about $p$ as a
  distribution
  \begin{itemize}
  \item Simple (probably wrong) approach: take the likelihood function
    to give the distribution conditional on $p$
  \item Model the uncertainty about $p$ as (for now) uniform(0,1)
  \item Update uncertainty to condition on the data set $f_P(p ∣ Y)$
  \item use Bayes's rule:
    \[f_P(p ∣ Y) = f_Y(y ∣ p) f_P(p) / f_Y(y)\]
  \item $f_Y(y) = ∫ f_Y(y ∣ p) f_P(p) dp$ (but usually ignore)
  \item so posterior is proportional to 
    \[∏ p_i^{y_i} (1 - p_i)^{1 - y_i}\] for $p ∈ [0,1]$ (draw curve)
  \item mode of the posterior is 3/4 still
  \item mean is 2/3 (so give 2:1 odds)
  \item can get ``credible'' intervals from posterior; point estimates
    from loss functions etc (34\% to 92\%)
  \item uniform isn't a great way to represent ``no information'', but
    is very common in econometrics (very problematic in TS)
  \item Jeffreys prior (1961)
  \item Reference prior: Bernardo (1979), Berger, Bernardo, and Sun
    (2009) See Phillips (1991) and Berger and Yang (1994)
  \item If you have and want to use good domain knowledge, you can
    specify an informative prior
  \end{itemize}
\item Simulation: MCMC (we won't cover), but basically you don't need
  to be able to directly draw from the posterior distribution as long
  as you can evaluate it
\end{itemize}

\subsection{So, IRFs through Bayesian estimators}

\begin{itemize}
\item specify prior for AR coefficients and VCV
\item Draw candidate values of the AR coefficients from the posterior
\item For each one, calculate the IRF as before
\item Summarize with mean, mode, or credible set
\item It is \emph{very unlikely} that these are uniformly valid
  credible sets if you use the uniform prior
\end{itemize}

\section{Basic ideas for IRF}

AR and MA models that we discussed earlier are \emph{descriptive}
models, but don't necessarily have any economic content
\[φ(L) y_t = θ(L) e_t\]
describes the second moments of $y_t$; the usual concerns about
simultaneity apply here (and even more forcefully than usual).

\begin{itemize}
\item Interested in dynamics
\item Want to know, what happens to $y_{t+1}, y_{t+2},…, y_{t+h},…$
  given an exogenous change to $y_{1,t}$?
\item What is an exogenous change to $y_{1,t}$? Can it be isolated
  from changes to $y_{2,t}$, $y_{3,t}$, etc?
\item Monetary shock
\item Technology shock
\item Financial shock (credit…)
\item Oil price shock
\item UNANTICIPATED
\end{itemize}

Look at $MA(∞)$ representation (assuming stationarity, etc) zero mean
for simplicity \[y_t = C(L) e_t\] where $C(L) = φ(L)^{-1} θ(L) e_t$

we want to transform/rotate $e_t$ so that we have ``structural''
shocks $u_t$,
\[u_t = V e_t\] typically $u_t$ is normalized so that
\[var( u_t ) = I\] (uncorrelation is kind of the definition of a macro
shock) and we have
\[y_t = C(L) V^{-1} u_t\ (rewrite\ as)\ B(L) u_t\]

So then
\[y_t = B_0 u_t + B_1 u_{t-1} + B_2 u_{t-2} + ⋯\]
and the effect of
\begin{itemize}
\item $u_{1t}$ shock to $y_t$ is $B_0 (1 0 0 … 0)$
\item $u_{1t}$ shock to $y_{t+1}$ is $B_1 (1 0 0 … 0)'$
\item etc.
\item $u_{1t}$ shock to $y_{t+k}$ is $B_k (1 0 0 … 0)'$
\end{itemize}
notice that $V$ is not identified and so the shocks are not identified:
if $W$ is any orthonormal matrix, so that $W W' = I$, then $W u_t$ has
variance $I$ as well, and
\[C(L) e_t = C(L) V^{-1} u_t = C(L) (VW)^{-1} ( W u_t )\]

\subsection{VAR representation}

(mention rank conditions)

\[B(L)^{-1} = A(L);\]
\begin{itemize}
\item structural form: $A_0 y_t = A_1 y_{t-1} + ⋯ + A_p y_{t-p} + u_t$
\item reduced form: $y_t = A_0^{-1} A_1 y_{t-1} + ⋯ + A_0^{-1} A_p
  y_{t-p} + A_0^{-1} u_t$
\end{itemize}
in this notation, the effect of
\begin{itemize}
\item $u_{1t}$ shock to $y_t$ is $A_0^{-1} (1 0 … 0)'$
\item $u_{1t}$ shock to $y_{t+1}$ is $A_0^{-1} A_1 × shocked\ y_t =
  A_0^{-1} A_1 A_0^{-1} (1 0 … 0)'$
\item $u_{1t}$ shock to $y_{t+2} = A_0^{-1} A_1 × shocked\ y_{t+1} +
  A_0^{-1} A_2 × shocked\ y_t = (A_0^{-1} A_1) × (A_0^{-1} A_1)
  A_0^{-1} (1 0 ⋯ 0)'$
  \begin{itemize}
  \item $(A_0^{-1} A_2) * A_0^{-1} (1 0 ⋯ 0)'$
  \end{itemize}
\end{itemize}
etc (these formulas are really easy if you use the canonical form of
the VAR)

\begin{itemize}
\item With OLS, we can consistently estimate $A_0^{-1} A_1$, $A_0^{-1}
  A_2$, etc and $A_0^{-1} A_0^{-1\prime}$ (i.e. the variance-covariance
  matrix of the reduced form error).
\item BUT, we actually need to know \emph{all} of the elements of
  $A_0$ or $A_0^{-1}$, and $A_0^{-1} A_0^{-1\prime}$ is positive definite
  and so has only $n (n-1) / 2$ unique elements, not the $n^2$.
\end{itemize}

\section{Identification}

\begin{itemize}
\item Short run identification: Constrain $A_0$ or $A_0^{-1}$, not
  $(A_0^{-1} A_i)$.
\item that would be the typical simultaneous equations approach, IV,
  etc
\item Long-run identification (we'll discuss in a bit) -- based on the
  idea that some shocks (productivity shocks for example) permanently
  affect the level of some variables (per-capita output, for example)
\item Partial identification
\end{itemize}

\subsection{Short-run identification}

Suppose we know that
\begin{itemize}
\item $y_{1t}$ ignores $u_{2t}$ through $u_{nt}$
\item $y_{2t}$ ignores $u_{3t}$ through $u_{nt}$
\item dot dot dot
\item $y_{n-1,t}$ ignores $u_{nt}$
\item $y_{n,t}$ responds to all of the shocks
\end{itemize}
Then you know that $A_0^{-1}$ is lower triangular and so it's
identified.

\begin{itemize}
\item Sounds sort of ridiculous, but this was Sims's (1980) first
  identification approach;
\item You could imagine that the Federal Reserve sees what's happened
  in the economy before acting, so it would be at the bottom; $u_{nt}$
  would represent a monetary policy shock (the equation for $y_{nt}$
  would be something like a Taylor rule and $y_{nt}$ needs to be an
  interest rate).
\item ie imagine inflation, unemployment, and interest rates
\item probably want to leave the other coefficients of the interest
  rate equation unspecified so that you're allowing it to be forward
  looking (or include additional forward looking variables)
\item could specify 
  \begin{equation*}
    R_t = R^* + coef * (inflation_t - inflation^*) + coef * (unemp_t - unemp^*) + coefs *
    R_{t-1} + ⋯ + e_{Rt}
  \end{equation*}
\item but now this equation needs to be right (if it is, another
  source of identification)
\item This has the ``advantage'' of being especially easy to calculate
  (comes from a Choleski decomposition of the sample covariance
  matrix).
\item This is kind of old-fashioned; you can't usually identify every
  shock this way;
\end{itemize}

\subsection{semi-structural}
\[y_t = (Δ gdp_t,\ inflation_t,\ fed\ funds_t)\]
``identify'' with recursive structure: 
\begin{itemize}
\item $Δ gdp_t$ only responds to
``first shock''
\item $inflation_t$ only responds to ``first shock'' and
``second shock''
\item $fed\ funds_t$ is set by monetary policy and responds
to everything
\item important thing is that it happens last
\item identify ``monetary policy shock'' as deviations from
  ``endogenous policy''
\item don't identify any other shocks (recursive structure for first
  two is really just a normalization device).
\end{itemize}

\begin{itemize}
\item This could be more convincing with higher-frequency data (ie
  kalman filter/ state space models) or better policy knowledge
\item An argument that prices respond more quickly than quantities
  seems plausible overall (remembering that the VAR is supposed to be
  taking care of the endogenous component of both)
\end{itemize}

\subsection{Kilian (2010) has other examples}

\begin{itemize}
\item Any economic argument to pin down particular values of
  $A_0^{-1}$ or $A_0$ can do the same trick.
\end{itemize}

\subsection{Long-run identification}

Introduced in Blanchard and Quah (1989 AER)

\begin{itemize}
\item Their model: bivariate VAR ($Δ gdp_t$, $u_t$)
\item $Δ gdp_t$ is I(0)
\item Assume for argument's sake that $gdp_0 = 0$
\item Assumption: want to identify the shocks that have a long-run
  effect on $gdp_t$ (they argue that these have interpretation as a
  supply shock)
\item Start with structural MA representation:
\item $v_t$ consists of the policy shocks; first element is ``supply''
  shock
  \[( Δ gdp_t, u_t ) = C(L) e_t = D(L) v_t\]
  Simple rewrite:
  \[ ∑_{s=1}^t (Δ gdp_s, u_s) = D(L) ∑_s v_s \]
\item Apply the ``beveridge-nelson decomposition'' (note that $D(1) =
  D_0 + D_1 + …$
  \begin{align*}
    ∑_{s=1}^t D(L) v_s
    &= ∑_{s=1}^t ∑_{j=0}^∞ D_j v_{s-j} \\
    &= ∑_{j=0}^∞ v_{t-j} ∑_{s=0}^j D_s \\
    &= (D(1) - ∑_{j=1}^∞ D_j) v_t + (D(1) - ∑_{j=2}^∞ D_j) v_{t-1} + ⋯ \\
    &= D(1) ∑_{s=1}^t v_s + D^*(L) v_t
  \end{align*}
  where $D_k^* = - ∑_{j=k+1}^∞ D_j$ and
  $D(1) = \begin{pmatrix} 
    D_{11}(1) & D_{12}(1) \\ D_{21}(1) & D_{22}(1)
  \end{pmatrix}$
\item This gets us $(gdp_t, ∑_{s=1}^t u_s) = sum = D(1) ∑_{s=1}^t v_s
  + D(L) v_t$
\item second part are purely transitory shocks
\item $D(1)$ part are the ``permanent shocks''
\item model makes no claim about permanent
  shock on summed unemployment
\item does make claim about shock on
  $gdp_t$
\item second element of $v_t$ can't have permanent effect, so
  $D_{12}(1)$ must be zero!
  \[vcv(v_s) = I vcv(C(1) e_t) = vcv(D(1) v_t),\]
  so
  \begin{itemize}
  \item $C(1) Σ C(1)`= D(1) D(1)'$ (and LHS can be estimated
    consistently);
  \item here $D(1)$ can be estimated as the choleski decomposition of
    estimated $C(1) Σ C(1)'$ (and know that, if we have the reduced
    form VAR model $Φ(L) y_t = e_t$ then $Φ(L)^{-1} = C(L)$ (and this
    holds for $L = 1$)
  \item So then $v_t = \hat D(1)^{-1} \hat C(1) e_t$ and $A_0^{-1} =
    \hat D(1)^{-1} \hat C(1)$
  \end{itemize}
\end{itemize}

%%% Local Variables: 
%%% mode: latex
%%% TeX-master: "../textbook"
%%% End: 