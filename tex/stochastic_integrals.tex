% Copyright © 2013-2014 Gray Calhoun

% Permission is granted to copy, distribute and/or modify this
% document under the terms of the GNU Free Documentation License,
% Version 1.3 or any later version published by the Free Software
% Foundation; with no Invariant Sections, no Front-Cover Texts, and no
% Back-Cover Texts.  A copy of the license is included in the file
% LICENSE.tex and is also available online at
% <http://www.gnu.org/copyleft/fdl.html>.

\section{Stochastic integration}

\subsection{Start}

\begin{itemize}
\item
  $W(\lambda)$ is a \emph{Weiner process} or \emph{Brownian Motion} if it is
  \begin{itemize}
  \item continuous and mean zero
  \item $W(t) - W(s) \sim N(0, t-s)$ for any $t$ and $s$
  \item Non-overlapping intervals are independent
  \end{itemize}
\item Draw diagram of sample paths w/ $\sqrt T$ envelope
\item Obviously, if $e_t$ is an MDS with variance 1, we have
  \[1/\sqrt{T} \sum_{t=1}^{[\lambda T]} e_t \to^d W(\lambda)\]
  under reasonable assumptions; want to be able to write this as an
  integral $\int_0^\lambda dW$

  \begin{itemize}
  \item
    $\lim 1/T \sum_{t=1}^{[\lambda T]}$ maps to $\int_0^\lambda dt$
  \item
    $\lim e_t \sqrt{T}$ must map to $dW/dt$

    \begin{itemize}
    \item $e_t/\sqrt{T} ≈ dW$
    \item i.e. $dW^2 ≈ dt$
    \item You can also see this from writing out
      \[dW/dt = \lim_{h \to 0} (W(t+h) - W(t)) / h\]
    \end{itemize}
  \item This is just intuition; proving is a little more difficult.
  \item Implies that $W(\lambda)$ is not differentiable a.e.
  \end{itemize}
\item We only need a few basic results; there are entire classes you
  can take on working with Ito integrals and SDEs
\end{itemize}

\subsection{Application \& use}

\begin{itemize}
\item Continuous mapping theorem: if $f$ is a continuous functional on
  $[0,1]$ then \[f(1/\sqrt{T} \sum_{t=1}^{[\lambda T]} e_t) \to^d f(W(\lambda))\]
  and, generally, if $g_t \to^d g$ where $g$ is a random process on
  $[0,1]$ then $f(g_t) \to^d f(g)$
\item Functional delta-method is similar
\item Also can use integration results
\end{itemize}

\subsection{Unit roots in regression}

\subsection{sub- Spurious regression}

Suppose that we have a multivariate unit root process
\[y_t = y_{t-1} + e_t\] and we run the regression
\[y_{1t} = \beta y_{2,t-1} + u_t\]

Define

\begin{itemize}
\item $v_1 = (1, 0)'$
\item $v_2 = (0, 1)'$
\end{itemize}

The OLS coefficients can be written as
\begin{equation}
  \begin{split}
    \hat \beta &= \Big(\sum_{t=2}^T v_2' y_{t-1} y_{t-1}' v_2\Big)^{-1}
    \sum_{t=2}^T v_2' y_{t-1} y_t' v_1 \\
    &= \Big(v_2' \sum_{t=2}^T y_{t-1} y_{t-1}' v_2 \Big)^{-1} \Big(
    1/T^2 v_2' \sum_{t=2}^T y_{t-1} y_{t-1}' v_1
    + 1/T^2 v_2' \sum_{t=2}^T y_{t-1} e_t v_1 \Big) \\
    &\to^d \Big(v_2' \Sigma^{1/2} \int_0^1 W(s) W(s)' ds \Sigma^{1/2}
    v_2\Big)^{-1} v_2' \Sigma^{1/2} \int_0^1 W(s) W(s)' ds \Sigma^{1/2} v_1
\end{split}
\end{equation}
This is interesting because it means that $\hat \beta$ is not consistent
(i.e.~doesn't converge to $\beta$). You can also show that $t$-stats don't
work (the statistic diverges so the test rejects with probability 1 in
the limit)

Relevant texts:
\begin{itemize}
\item \citet{GN74}
\item \citet{Phi86}
\end{itemize}

\subsection{Regress I(1) on I(0)}

Take the last example, but suppose that we include a covariance
stationary I(0) regressor $x_{t-1}$ (but this is going to work for any
collection of I(0) regressors as well),
\[y_t = \beta_0 x_{t-1} + \beta_1 + \beta_2 y_{t-1} + u_t\] and say we want to get
limiting distributions for OLS. Without loss of generality, assume that
$E x_t = 0$; otherwise rewrite the equation as
\[y_t = \beta_0 (x_{t-1} - E x_{t-1}) + (\beta_1 + \beta_0 E x_{t-1}) + \beta_2 y_{t-1} + u_t\]
Also assume that $\var e_t$ is 1 to keep the notation as simple as
possible.

How to get asymp dist of $\hat \beta - \beta$? Key thing is that the different
elements are going to converge at different rates; let
\[\Lambda = diag(\sqrt{T}, \sqrt{T}, T)\] so
\begin{equation}
\begin{split}
  \Lambda (\hat \beta - \beta) &=  \left( \Lambda^{-1} \sum_{t=2}^T
    \begin{pmatrix}
      x_{t-1}^2       & x_{t-1}   & x_{t-1} y_{t-1} \\
      x_{t-1}         & 1         & y_{t-1} \\
      x_{t-1} y_{t-1} & y_{t-1}   & y_{t-1}^2
    \end{pmatrix} \Lambda^{-1} \right)^{-1}
  \Lambda^{-1} \sum_{t=2}^T
  \begin{pmatrix}
    x_{t-1} e_t \\ e_t \\ y_{t-1} e_t
  \end{pmatrix}\\
  & \to^d
  \begin{pmatrix}
    E x_t^2 & 0             & 0 \\
    0       & 1             & \int_0^1 W(s) ds \\
    0       & \int_0^1 W(s) ds & \int_0^1 W(s)^2 ds
  \end{pmatrix}^{-1}
  \begin{pmatrix} (E x_t^2)^{1/2} W(1) \\ W(1) \\ \int_0^1 W(s) dW(s) \end{pmatrix}
\end{split}
\end{equation}
Where the 0 terms in the X'X matrix come from
\[T^{-1} \sum_{t=2}^T x_{t-1} \to^p 0\] from the LLN and
\[T^{-1} \sum_{t=2}^T x_{t-1} y_{t-1} \to^d \int_0^1 W_y(s) dW_x(s)\] so
\[T^{-3/2} \sum_{t=2}^T x_{t-1} y_{t-1} \to^p 0.\]

Since the $X'X$ component is block diagonal, the asymptotic distribution
of the first element of $\hat \beta$ is $(E x_t^2)^{-1} W(1)$, making it
consistent and asymptotically normal at the usual rate. The coefficient
on the I(1) term and on the constant are both not Normal.

Note that I got a little too cute in class and forgot that the constant
term is \emph{not} I(0); note that $T^{-1} \sum_t 1$ obeys a ``LLN'' and
converges to 1, but $T^{-½} \sum_t 1$ definitely does not obey a ``CLT''.
So the estimator on $x_{t-1}$ is asymptotically normal, but the
estimator of the intercept is not.

\textbf{Additional lag structure}

Suppose now you run the regression
\[y_t = \beta_0 + \beta_1 y_{t-1} + \beta_2 y_{t-2} + u_t\]

\begin{itemize}
\item know that OLS estimator of $\beta_0$ is normal and correctly centered
\item we can rewrite the relationship as
  \[y_t = \beta_0 + \beta_1 \Delta y_{t-1} + (\beta_2 + \beta_1) y_{t-2} + u_t\] and
  estimating $\beta_1$ in this equation will give

  \begin{itemize}
  \item A numerically identical estimate as in the previous equation
  \item A consistent and asymptotically normal estimator of $\beta_1$
  \item Note that our estimate of $\beta_1 + \beta_2$ will have an awkward
    distribution
  \item So the OLS estimate of $\beta_1$ in the original regression is
    consistent and asymptotically normal
  \end{itemize}
\item Similarly, we can show that the OLS estimate of $\beta_2$ in the
  original regression is consistent and asymptotically normal.
\item Note that the estimate of $\beta$ is not jointly normal, since
  $\beta_1+\beta_2$ has a non-normal distribution.
\item This is true whenever you can rewrite the expressions so that
  coefficients appear on I(0) components and has implications for
  cointegration.
\end{itemize}

Relevant texts:
\begin{itemize}
\item \citet{SSW90}
\end{itemize}

\subsection{Regress I(0) on I(1)}
To be added\ldots

\subsection{Additional references to add from syllabus}
\begin{itemize}
\item \citet{Wa94}
\item \citet{Da94}%
\footnote{Comprehensive overview of probability theory and asymptotic
  results for dependent data. Some of the CLTs and LLNs have been improved
  on since this book was published, often by Davidson and his coauthors.} %
\item \citet{Ga97}%
\footnote{Introductory text for probability theory focused on econometrics;
  I have assigned it in Econ 671 some years. Great to review results you
  should know but may have forgotten.} %
\item \citet{Wh01}%
\footnote{Introduction, but focuses on OLS and IV and extends coverage
  to dependent data.} %
\end{itemize}

%%% Local Variables:
%%% mode: latex
%%% TeX-master: "../macroeconometrics"
%%% End:
