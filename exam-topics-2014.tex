\documentclass[nofonts,nols]{tufte-handout}
\title{Final exam topics, Econ 674}
\author{Helle Bunzel and Gray Calhoun}
\date{Fall semester, 2014}

\morefloats
\usepackage{
  booktabs,
  enumitem,
  fancyvrb,
  fancyhdr,
  marginfix,
  tabularx,
  url,
  verbatim,
  xcolor,
}
\usepackage[T1]{fontenc}
\usepackage[utf8x]{inputenc}
\usepackage[bitstream-charter]{mathdesign}
\usepackage[letterspace=35]{microtype}
\usepackage[osf]{sourcecodepro}
\usepackage{sourcesanspro}

% Change to ragged-right without inline math breaks; this helps avoid
% awkward line breaks or extensions into the right margin in math or
% code.
\RaggedRight
\relpenalty=10000
\binoppenalty=10000
\setlength{\parindent}{1em}

\renewcommand{\smallcaps}[1]{\allcaps{#1}}
\renewcommand{\allcaps}[1]{\textls{\MakeUppercase{#1}}}

\renewcommand{\FancyVerbFormatLine}[1]{\hspace{\leftmargini}#1}
\fvset{frame=leftline,rulecolor=\color{lightgray}}

\urlstyle{same}
\DeclareUrlCommand\url{%
\def\UrlLeft{\guillemotleft}%
\def\UrlRight{\guillemotright}%
\urlstyle{same}}

\DisableLigatures{family = tt*}
\frenchspacing

\newcommand{\email}[1]{\href{mailto:#1}{\nolinkurl{#1}}}
\newcommand{\ARMA}{\allcaps{ARMA}}
\newcommand{\DSGE}{\allcaps{DSGE}}
\newcommand{\HAC}{\allcaps{HAC}}
\newcommand{\OH}{\allcaps{OH}}
\newcommand{\SVAR}{\allcaps{SVAR}}
\newcommand{\TA}{\allcaps{TA}}
\newcommand{\VAR}{\allcaps{VAR}}

\renewcommand{\labelitemi}{{$\circ$}}
\renewcommand{\labelitemii}{\footnotesize$\circ$}
\renewcommand{\labelitemiii}{\textperiodcentered}
\renewcommand{\labelitemiv}{\footnotesize\textperiodcentered}

\begin{document}
\maketitle

\begin{table*}[h]
  \begin{tabularx}{\textwidth}{rXX}
    \toprule
        & Exam topic                                                             & Suggsested reading                                                                                      \\
    \midrule
    1.  & The ``Lucas Critique''                                                 & \citep{Lu76}, \citep{No11}, \citep{Si12}                                                                \\
    2.  & Stationary univariate \ARMA\ models                                    & \citep{Ha94} Ch 3 -- 5                                                                                  \\
    3.  & Identification in \SVAR s                                              & \citep{Si80}, \citep{Ki13}                                                                              \\
    4.  & Statistical aspects of stationary \VAR s                               & \citep{Ha94} Ch 10 and 11                                                                               \\
    5.  & Spectral representation                                                & \citep{Ha94} Ch 6                                                                                       \\
    6.  & Filtering                                                              & \citep{Ha94} Ch 6, \citep{BK99}                                                                         \\
    7.  & \HAC\ Estimation                                                       & \citep{Ha94} 10.5, \citep{NW87}, \citep{An91}                                                           \\
    8.  & Testing using fixed bandwidth asymptotics                              & \citep{KV05}, \citep{Su14}                                                                              \\
    9.  & Univariate \ARMA\ processes with stochastic and non-stochastic trends  & \citep{Ha94} Ch 15 -- 17                                                                                \\
    10. & Testing for a break in the simple regression model with i.i.d.\ errors & \citep[Sec 5]{St94}                                                                                     \\
    11. & Testing for breaks in realistic time series models                     & \citep{An03}, \citep{Ba97}, \citep{BP98}                                                                \\
    12. & Bootstrapping i.i.d.\ data                                             & \citep{Ha14} Ch 10                                                                                      \\
    13. & Bootstrapping time series data                                         & B\"{u}hlmann (1999) research report (on Box), \citep{HHK03} (more theory than covered in class; on Box) \\
    14. & Determining the trend structure of univariate time series              & \citep{AB00}, \citep{HLT07}                                                                             \\
    15. & Nonstationary \VAR s (without cointegration)                           & \citep{Ha94} Ch 18                                                                                      \\
    16. & Estimating and testing cointegrating relationships                     & \citep{Ha94} Ch 19,  \citep{Jo14}                                                                       \\
    17. & Error Correction Models (estimation and inference)                     & \citep{Ha94} Ch 20, \citep{Da13}, \citep{Jo14}                                                          \\
    18. & Multiple hypothesis testing                                            & \citep{RW05}, \citep{Ro08}                                                                              \\
    19. & Bayesian inference                                                     & \citep{Ha94} Ch 12, \citep{Ch12}                                                                        \\
    20. & State space models (focusing on Bayesian estimation)                   & \citep{Ha94} Ch 12 and 13, \citep{Fe11}, \citep{Ch12}                                                   \\
    \bottomrule
  \end{tabularx}
\end{table*}

\begin{fullwidth}
\noindent%
The recommended resources listed on the syllabus --- \citep{Lu06},
\citep{Ca07}, \citep{MS08}, and \citep{SW08} --- also have a lot of
useful material on these topics. You also own several econometrics
books from the core sequence and they cover some of this material as
well.

The ``suggested reading'' is meant to be an \emph{exhaustive} and
\emph{comprehensive} list: we could not imagine any student reading
all of it in detail. (Feel free to take that as a challenge!) You are
expected to pick out a subset or overview that you find interesting
and/or important and prepare to present that material.

\bibliographystyle{alpha}
\bibliography{tex/references}
\end{fullwidth}
\end{document}
